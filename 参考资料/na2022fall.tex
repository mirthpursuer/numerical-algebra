\documentclass{article}
\usepackage[UTF8]{ctex}
\usepackage{float,indentfirst,verbatim,fancyhdr,graphicx,listings,longtable,amsmath, amsfonts,amssymb}
\usepackage{algorithm}
\usepackage{amsmath}
\usepackage{algpseudocode}
\usepackage{float}
\usepackage{longtable}
\usepackage{lscape}
\usepackage{colortbl}
\usepackage{array}
\usepackage{multirow}
\textheight 23.5cm \textwidth 15.8cm
%\leftskip -1cm
\topmargin -1.5cm \oddsidemargin 0.3cm \evensidemargin -0.3cm
\usepackage[framemethod=TikZ]{mdframed}
\usepackage{url}   % 网页链接
\usepackage{subcaption} % 子标题
%\usepackage[a3paper, margin=2cm]{geometry}
\usepackage[left=2.50cm, right=2.50cm, top=2.50cm, bottom=2.50cm]{geometry} %页边距
\usepackage{helvet}
\usepackage{amsmath, amsfonts, amssymb} % 数学公式、符号
%\usepackage[english]{babel}
\usepackage{graphicx}   % 图片
\usepackage{url}        % 超链接
\usepackage{bm}         % 加粗方程字体
\usepackage{multirow}
\usepackage{booktabs}
%\usepackage{algorithm}
%\usepackage{algorithmic}
\usepackage{esint}
\usepackage{hyperref} %bookmarks
\usepackage{fancyhdr} %设置页眉、页脚
%\hypersetup{colorlinks, bookmarks, unicode} %unicode
\usepackage{multicol}
\usepackage{graphicx}
\usepackage{xcolor}
\title{数值代数2022秋期末试题}
\author{2022年12月9日}
\date{}
\begin{document}
	\maketitle
一.(10分) 已知
$$
A\in C^{n\times n},||A||_*<1,||I||_*=1
$$

证明

1. 对所有的范数,都有

$$
\begin{aligned}\rho(A)\leq||A||\end{aligned}
$$

2.

$$||(I-A)^{-1}-\sum_{k=0}^mA^k||_*\leq\frac{||A||_*^{m+1}}{1-||A||_*}$$

注记:都是书上的简单定理 (定义课本上证明长度不超过半页纸的定理是简单定理).

二.(10分)
$$
\begin{aligned}A\in R^{n\times n},A=4I+2J_n(0)-J_n(0)^T\end{aligned}
$$

证明:在 $w\in(0,1)$ 的时候 SOR 迭代法是收敛的。

注记:本质是证明对角严格占优阵的 SOR 收敛定理,属于书上简单定理的证明.

三.(20分)

$$A=\begin{pmatrix}\frac{11}{5}&\frac{-4}{5}&\frac{2}{5}&\frac{-2}{5}\\\frac{-2}{5}&\frac{8}{5}&\frac{1}{5}&\frac{-1}{5}\\-2&2&1&-4\\\frac{-18}{5}&\frac{12}{5}&\frac{-6}{5}&\frac{-9}{5}\end{pmatrix},X=\begin{pmatrix}1&0&-1&0\\0&1&-1&0\\1&2&1&1\\-1&0&0&1\end{pmatrix}$$

证明:

1. $X^{-1}AX$ 是对角阵

2. 写出 $A$ 用幂法作用在初始向量后得到的向量序列,并且证明 $X^{-1}A$没有 $0$ 分量的时候,序列存在两个收敛子列

3. 说出上述收敛子列极限和 $A$ 的特征向量的关系

注记:如果熟悉幂法的话,自然不难知道迭代向量列长什么样。(事实上,课本证明幂法收敛定理的时候就做了和这题类似的操作)

四.(15 分) 


$A$是$r$个$Household$变换的乘积,证明有分解式$A=I-WV^T$其中$ W,V $都是$ n×r$ 的矩阵 

注记:一道比较考验线性代数功夫的题目,需要一定的手法。

五. (10 分)

$H$ 是不可约的上 $Hesse$ 阵,证明存在对角阵把他相似到下次对角元都是 1 的上$ Hesse $阵

注记:一道有手就行的题目。

六.  (20 分)

课本第一章 16 题

注记:不算特别困难的作业题,至少不是解题中特别多不平凡步骤的作业题。

七.(15分)

$ A $有$ k$ 个互异的特征值,是 $n$ 阶方阵,$r$ 是 $n$ 阶向量,证明
$$
\begin{aligned}dim(span(r,Ar,\cdots,A^{n-1}r))\leq k\end{aligned}
$$

注记:不算特别困难的作业题,有线代功夫就可以做。

\end{document}