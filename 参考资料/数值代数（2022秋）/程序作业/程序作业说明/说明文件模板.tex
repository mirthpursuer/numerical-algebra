\documentclass{article}
\textheight 23.5cm \textwidth 15.8cm
%\leftskip -1cm
\topmargin -1.5cm \oddsidemargin 0.3cm \evensidemargin -0.3cm
%\documentclass[final]{siamltex}

\usepackage{ctex}
\usepackage{verbatim}
\usepackage{fancyhdr}
\usepackage{graphicx}
\usepackage{amsmath}
\usepackage{amssymb}
\usepackage{float}
\usepackage{multirow}
\usepackage{colortbl}
\usepackage{amsthm}

\newcommand{\df}{\mathrm{d}f}
\newcommand{\du}{\mathrm{d}u}
\newcommand{\dv}{\mathrm{d}v}
\newcommand{\dx}{\mathrm{d}x}
\newcommand{\dy}{\mathrm{d}y}
\newcommand{\dz}{\mathrm{d}z}
\newcommand{\va}{\boldsymbol{a}}
\newcommand{\vb}{\boldsymbol{b}}
\newcommand{\vx}{\boldsymbol{x}}
\newcommand{\vy}{\boldsymbol{y}}
\newcommand{\vp}{\boldsymbol{p}}
\newcommand{\vq}{\boldsymbol{q}}
\newcommand{\vr}{\boldsymbol{r}}
\newcommand{\vn}{\boldsymbol{n}}
\newcommand{\vf}{\boldsymbol{f}}
\newcommand{\vg}{\boldsymbol{g}}

%\pagestyle{fancy} \lhead{} \chead{}
%\rhead{}
%
%\lfoot{} \cfoot{} \rfoot{\thepage}
%\renewcommand{\headrulewidth}{0.4pt}
%\renewcommand{\footrulewidth}{0.4pt}
%
\title{数值代数程序说明}
\author{张学静 BA22001010}

\begin{document}
\maketitle

\section{提交说明}

1.将压缩包提交至邮箱ustcszds2020@163.com,每章提交一次,ddl会视情况另行通知,请关注群内公告。

2.尽量使用校内邮箱发送邮件。

3.文件名+邮件主题格式:第1/2组 姓名 学号(如:第1组 张学静 BA22001010)。

4.打包内容为该说明文件中包含的除.tex文件外的所有文件,如图\ref{filelist},不要包含跑程序时出现的.vs或debug文件夹(提交的文件大小应该很小,可以以此自查)。
\begin{figure}[H]
\label{filelist}
\centering %使图片居中
\includegraphics[width=5cm]{1.png}
\caption{打包内容} %图片名称
\end{figure}

\section{程序作业说明}

1.程序作业通常为课本上的上机习题,老师可能会加一些额外的要求。

2.按要求完成习题内容,并在代码中添加必要的注释。其余在报告中体现。

3.可以参考讨论但是严禁copy前几届或是同学的代码,应老师要求,一经发现本次作业0分处理。

4.给大家建了第一次作业的框架,可以参考,也可以自己写一个更好的。

5.不要上交跑不出来的程序。

6.不建议将向量打印成列向量。

\section{报告说明}

1.报告需要包含问题描述,程序运行结果和结果分析等内容。

2.上交的报告应为pdf文件(markdown和word等需要转化成pdf)。

3.报告中的程序运行结果最好是截图,不要直接复制文本。

4.若用latex写报告,可用该文档的模板,见.tex文件。

\end{document}