\documentclass{article}
\usepackage[UTF8]{ctex}
\usepackage{float,indentfirst,verbatim,fancyhdr,graphicx,listings,longtable,amsmath, amsfonts,amssymb}
\usepackage{algorithm}
\usepackage{amsmath}
\usepackage{algpseudocode}
\usepackage{float}
\usepackage{longtable}
\usepackage{lscape}
\usepackage{colortbl}
\usepackage{array}
\usepackage{multirow}
\textheight 23.5cm \textwidth 15.8cm
%\leftskip -1cm
\topmargin -1.5cm \oddsidemargin 0.3cm \evensidemargin -0.3cm
\usepackage[framemethod=TikZ]{mdframed}
\usepackage{url}   % 网页链接
\usepackage{subcaption} % 子标题
%\usepackage[a3paper, margin=2cm]{geometry}
\usepackage[left=2.50cm, right=2.50cm, top=2.50cm, bottom=2.50cm]{geometry} %页边距
\usepackage{helvet}
\usepackage{amsmath, amsfonts, amssymb} % 数学公式、符号
%\usepackage[english]{babel}
\usepackage{graphicx}   % 图片
\usepackage{url}        % 超链接
\usepackage{bm}         % 加粗方程字体
\usepackage{multirow}
\usepackage{booktabs}
%\usepackage{algorithm}
%\usepackage{algorithmic}
\usepackage{esint}
\usepackage{hyperref} %bookmarks
\usepackage{fancyhdr} %设置页眉、页脚
%\hypersetup{colorlinks, bookmarks, unicode} %unicode
\usepackage{multicol}
\usepackage{graphicx}
\usepackage{xcolor}
\title{数值代数2023秋期末试题}
\author{2022年12月21日}
\date{}
\begin{document}
	\maketitle
	
1.(10分)

设 $B$ 是 $A$ 的任意子矩阵,且是方阵,证明 $\|B\|_p\leq\|A\|_p.\|\cdot\|_p$ 表示相应矩阵由对应尺寸向量的 $p$ 范数诱导粗的矩阵算子范数,$1\leq p\leq\infty.$ 

2.(15分)

(1) 对于给定的单位向量 x, 构造两个不同的正交矩阵 $Q_1,Q_2$ 使得 $Q_ie_1=x,i=1,2$ 

(2) 设 $A\in\mathbf{C}^{n\times n}$,并假定 $\lambda\in\mathbf{C}u\in\mathbf{C}^n(u\neq0)$, 且 λ不是 A 的特征值。

证明:可选择 $E\in\mathbf{C}^{n\times n}$满足 $||E||_F=\frac{||u||_2}{||v||_2}$
使得向量 $v=(\lambda I-A)^{-1}u$ 是 A + E 的一个特征向量。


3.(20分)

(1) 证明矩阵单特征值的左右特征向量不垂直.

(2) 证明对称矩阵不同特征值对应的特征向量互相垂直. 

(3) 设

$$\left.A=\left(\begin{array}{cccc}1&3&0&0\\0&1&2&0\\0&0&1&1\\0&0&0&1\end{array}\right.\right)$$

考察对 $u_0=(1,1,1,1)^T$ 应用幂法所得序列的特性,并给出得到精确到三位有效数字所需的迭代次数。

4.(15分)

$$T_n=\begin{pmatrix}\alpha_1&\beta_2&0&0&\cdots&0\\\beta_2&\alpha_2&\beta_3&0&\cdots&0\\0&\beta_3&\alpha_3&\beta_4&\cdots&0\\\vdots&&\ddots&\ddots&\ddots&\\0&\cdots&0&\beta_{n-1}&\alpha_{n-1}&\beta_n\\0&0&\cdots&0&\beta_n&\alpha_n\end{pmatrix}$$

为实对称不可约三对角阵,设 $p_i(\lambda)$ 为 $T_n-\lambda I$ 的各阶顺序主子式,$i=1,2,\cdots,n.$

(a) 证明 $p_i(\lambda),p_{i+1}(\lambda)$ 没有公共根; 

(b) 证明 $p_n(\lambda)$ 只有单根。

\newpage
5.(25分)

$$A=\begin{pmatrix}\alpha_1&\beta_2&0&0&\cdots&0\\\beta_2&\alpha_2&\beta_3&0&\cdots&0\\0&\beta_3&\alpha_3&\beta_4&\cdots&0\\\vdots&&\ddots&\ddots&\ddots&\\0&\cdots&0&\beta_{n-1}&\alpha_{n-1}&\beta_n\\0&0&\cdots&0&\beta_n&\alpha_n\end{pmatrix}$$


为非奇异三对角阵,令 A=D+L+U,D 为 A 的对角部分,L 为 A 的下三角部分,U 为 A 的上三角部分。

(1)$B_J=I_n-D^{-1}A,B_{GS}=-(L+D)^{-1}U,p_{B_J}(\lambda),p_{B_{GS}}(\lambda)$ 分别为 $B_J,B_{GS}B_J,B_{GS}$ 的特征多项式

证明 
$$p_{B_J}(\lambda)=det(-D^{-1})det(L+\lambda D+U)$$ $$\begin{aligned}p_{B_{GS}}(\lambda)=det(-(L+D)^{-1})det(\lambda L+\lambda D+U)\end{aligned}$$

(2) 证明 $$det(\lambda^2L+\lambda^2D+U)=\lambda^ndet(L+\lambda D+U)$$

(3) 证明 $$\rho(B_{GS})=\rho(B_J)^2$$ 其中 $\rho(B_{GS}),\rho(B_J)$ 分别为 $B_{GS}$ 和 $B_J$ 的谱半径。当两种算法均收敛时,Jacobi 迭代和 G-S 迭代哪种收敛速度更快。并解释原因。

6.(15分)

 给定对称正定矩阵 $A$,如果 $A$ 至多有 $l$ 个互不相同的特征值,则共轭梯度法至多 $l$ 步就可以得到方程组 $Ax=b$ 的精确解。



\end{document}