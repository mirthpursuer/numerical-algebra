\documentclass{article}
\usepackage[UTF8]{ctex}
\usepackage{float,indentfirst,verbatim,fancyhdr,graphicx,listings,longtable,amsmath, amsfonts,amssymb}
\usepackage{algorithm}
\usepackage{amsmath}
\usepackage{algpseudocode}
\usepackage{float}
\usepackage{longtable}
\usepackage{lscape}
\usepackage{colortbl}
\usepackage{array}
\usepackage{multirow}
\textheight 23.5cm \textwidth 15.8cm
%\leftskip -1cm
\topmargin -1.5cm \oddsidemargin 0.3cm \evensidemargin -0.3cm
\usepackage[framemethod=TikZ]{mdframed}
\usepackage{url}   % 网页链接
\usepackage{subcaption} % 子标题
%\usepackage[a3paper, margin=2cm]{geometry}
\usepackage[left=2.50cm, right=2.50cm, top=2.50cm, bottom=2.50cm]{geometry} %页边距
\usepackage{helvet}
\usepackage{amsmath, amsfonts, amssymb} % 数学公式、符号
%\usepackage[english]{babel}
\usepackage{graphicx}   % 图片
\usepackage{url}        % 超链接
\usepackage{bm}         % 加粗方程字体
\usepackage{multirow}
\usepackage{booktabs}
%\usepackage{algorithm}
%\usepackage{algorithmic}
\usepackage{esint}
\usepackage{hyperref} %bookmarks
\usepackage{fancyhdr} %设置页眉、页脚
%\hypersetup{colorlinks, bookmarks, unicode} %unicode
\usepackage{multicol}
\usepackage{graphicx}
\usepackage{xcolor}
\title{数值代数实验报告}
\author{PB21010483 郭忠炜}

\begin{document}
\maketitle

\section*{\centerline{一. 问题描述}}

实现课本7.6节的SVD算法, 对附件svddata.txt中的矩阵进行奇异值分解,并按大小顺序列出所有奇异值。同时,计算出$PP^T-I, QQ^T-I, P\Sigma Q-A$ 的绝对值最大的元素,依次用ep, eq, et 表示。输出迭代次数和所得到的奇异值以及ep, eq, et。

%\newpage
\section*{\centerline{二. 程序介绍}}

\subsubsection*{生成整数矩阵}
\begin{itemize}
	\item \textbf{函数描述:} \texttt{intMatrix} 函数用于生成指定大小的整数矩阵。
	\item \textbf{使用方式:} 调用 \texttt{intMatrix(rows, cols)} 函数,传入所需的行数 \texttt{rows} 和列数 \texttt{cols},函数将生成一个以1开始连续递增的整数矩阵。
\end{itemize}

\subsubsection*{生成随机数矩阵 (\texttt{randomMatrix})}
\begin{itemize}
	\item \textbf{函数描述:} \texttt{randomMatrix} 函数用于生成具有指定行数和列数的随机数矩阵。
	\item \textbf{使用方式:} 调用 \texttt{randomMatrix(rows, cols)} 函数,传入所需的行数 \texttt{rows} 和列数 \texttt{cols},函数将生成一个包含随机数的矩阵,范围在1.0到10.0之间(可根据需要调整范围)。
\end{itemize}

\subsubsection*{生成单位阵 (\texttt{idMatrix})}
\begin{itemize}
	\item \textbf{函数描述:} \texttt{idMatrix} 函数用于生成指定阶数的单位阵(对角线为1,其余为0的方阵)。
	\item \textbf{使用方式:} 调用 \texttt{idMatrix(n)} 函数,传入所需的阶数 \texttt{n},函数将生成一个大小为 $n \times n$ 的单位阵。
\end{itemize}

\subsubsection*{实现矩阵的转置 (\texttt{transpose})}
\begin{itemize}
	\item \textbf{函数描述:} \texttt{transpose} 函数用于实现给定矩阵的转置操作。
	\item \textbf{使用方式:} 调用 \texttt{transpose(A)} 函数,传入需要转置的矩阵 \texttt{A},函数将返回其转置后的矩阵。
\end{itemize}

\subsubsection*{生成Household矩阵 (\texttt{Householder})}
\begin{itemize}
	\item \textbf{函数描述:} \texttt{Householder} 函数用于生成Householder矩阵,基于给定的向量和beta值。
	\item \textbf{使用方式:} 调用 \texttt{Householder(v, beta)} 函数,传入一个向量 \texttt{v} 和一个beta值,函数将返回相应的Householder矩阵。
\end{itemize}

\subsubsection*{实现m*n阶矩阵的二对角化 (\texttt{bidiag})}
\begin{itemize}
	\item \textbf{函数描述:} \texttt{bidiag} 函数用于将给定的 $m \times n$ 阶矩阵二对角化。
	\item \textbf{使用方式:} 调用 \texttt{bidiag(A, P, Q)} 函数,传入一个矩阵 \texttt{A} 以及两个空矩阵 \texttt{P} 和 \texttt{Q},函数将对矩阵 \texttt{A} 进行二对角化,并将正交矩阵结果存储在 \texttt{P} 和 \texttt{Q} 中。
\end{itemize}

\subsubsection*{将矩阵中满足条件的对角元和次对角元置零 (\texttt{zeroing2})}
\begin{itemize}
	\item \textbf{函数描述:} \texttt{zeroing2} 函数用于将满足条件的矩阵对角元和次对角元置零。
	\item \textbf{使用方式:} 调用 \texttt{zeroing2(A, epsilon)} 函数,传入一个矩阵 \texttt{A} 和一个阈值 \texttt{epsilon},函数将对 \texttt{A} 中满足条件的对角元和次对角元进行置零操作。
\end{itemize}

\subsubsection*{找到最大的p和最小的q (\texttt{pqq})}
\begin{itemize}
	\item \textbf{函数描述:} \texttt{pqq} 函数用于确定矩阵中最大的 $p$ 和最小的 $q$。
	\item \textbf{使用方式:} 调用 \texttt{pqq(B, p, q)} 函数,传入矩阵 \texttt{B} 和引用类型的 \texttt{p} 和 \texttt{q},函数将确定并存储最大的 $p$ 和最小的 $q$ 值。
\end{itemize}

\subsubsection*{计算得到对于a和b的cos与sin值 (\texttt{givens})}
\begin{itemize}
	\item \textbf{函数描述:} \texttt{givens} 函数用于计算给定的 $a$ 和 $b$ 对应的 $\cos$ 和 $\sin$ 值。
	\item \textbf{使用方式:} 调用 \texttt{givens(a, b, c, s)} 函数,传入两个浮点数 \texttt{a} 和 \texttt{b},以及引用类型的 \texttt{c} 和 \texttt{s},函数将计算并存储相应的 $\cos$ 和 $\sin$ 值。
\end{itemize}

\subsubsection*{由cos与sin值生成对应的Givens变换矩阵 (\texttt{Givens})}
\begin{itemize}
	\item \textbf{函数描述:} \texttt{Givens} 函数用于根据给定的 $\cos$ 和 $\sin$ 值生成相应的 Givens 变换矩阵。
	\item \textbf{使用方式:} 调用 \texttt{Givens(n, p, q, c, s)} 函数,传入矩阵大小 \texttt{n}、行列索引 \texttt{p} 和 \texttt{q},以及 $\cos$ 和 $\sin$ 值 \texttt{c} 和 \texttt{s},函数将生成对应的 Givens 变换矩阵。
\end{itemize}

\subsubsection*{若B22有对角元为0,用Givens变换把该行打成0 (\texttt{GivensSVD})}
\begin{itemize}
	\item \textbf{函数描述:} \texttt{GivensSVD} 函数根据条件对矩阵进行 Givens 变换操作。
	\item \textbf{使用方式:} 调用 \texttt{GivensSVD(B, B22, p, i, U)} 函数,传入矩阵 \texttt{B}、\texttt{B22}、整数 \texttt{p} 和 \texttt{i},以及矩阵 \texttt{U},函数将根据条件对矩阵进行 Givens 变换。
\end{itemize}

\subsubsection*{按照B22的维数进行对应的带Wilkinson位移的SVD迭代 (\texttt{WilkSVD})}
\begin{itemize}
	\item \textbf{函数描述:} \texttt{WilkSVD} 函数按照给定条件执行带 Wilkinson 位移的 SVD 迭代。
	\item \textbf{使用方式:} 调用 \texttt{WilkSVD(B, U, V)} 函数,传入矩阵 \texttt{B}、\texttt{U} 和 \texttt{V},函数将根据条件执行 SVD 迭代。
\end{itemize}

\subsubsection*{返回矩阵绝对值最大的元素 (\texttt{absmax})}
\begin{itemize}
	\item \textbf{函数描述:} \texttt{absmax} 函数用于返回矩阵中绝对值最大的元素。
	\item \textbf{使用方式:} 调用 \texttt{absmax(A)} 函数,传入矩阵 \texttt{A},函数将返回矩阵中绝对值最大的元素。
\end{itemize}

\subsubsection*{从小到大输出奇异值 (\texttt{singular})}
\begin{itemize}
	\item \textbf{函数描述:} \texttt{singular} 函数用于按照从小到大的顺序输出矩阵的奇异值。
	\item \textbf{使用方式:} 调用 \texttt{singular(A)} 函数,传入矩阵 \texttt{A},函数将输出矩阵的奇异值,按照从小到大的顺序。
\end{itemize}

\section*{\centerline{三. 实验结果}}

\begin{table}[H]
	\centering
	\begin{tabular}{|c|c|c|c|}
		\hline
		\multicolumn{4}{|c|}{奇异值(从小到大排序)} \\
		\hline
		0.375993 & 0.703989 & 0.880006 & 1.14018 \\
		\hline
		1.89863 & 2.60205 & 3.1445 & 4.98101 \\
		\hline
		5.94702 & 8.66648 & 32.2979 & 214.31 \\
		\hline
		迭代次数:18 & ep = 1.33227e-15 & eq = 1.11022e-15 & et = 3.26781e-05 \\
		\hline
	\end{tabular}
\end{table}



\section*{\centerline{四. 结果分析}}

这次实验应用奇异值分解的QR方法实现了对矩阵$A$的奇异值分解,在18次迭代计算得到矩阵的全部奇异值,并且迭代的精度参数$ep, eq, et$都控制在了很小的误差范围内,展现了算法的计算精度和稳定性。

\end{document}