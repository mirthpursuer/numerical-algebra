\documentclass{article}
\usepackage[UTF8]{ctex}
\usepackage{float,indentfirst,verbatim,fancyhdr,graphicx,listings,longtable,amsmath, amsfonts,amssymb}

\textheight 23.5cm \textwidth 15.8cm
%\leftskip -1cm
\topmargin -1.5cm \oddsidemargin 0.3cm \evensidemargin -0.3cm

%\pagestyle{fancy} \lhead{FDM Homework Template} \chead{}
%\rhead{\bfseries}
%
%\lfoot{} \cfoot{} \rfoot{\thepage}
%\renewcommand{\headrulewidth}{0.4pt}
%\renewcommand{\footrulewidth}{0.4pt}
%
\title{homework8}
\author{游瀚哲}

\begin{document}
\maketitle

\section*{一、作业要求}

参考课本 7.6.2 节 (P234-240)SVD 迭代完成 SVD 算法 7.6.3, 并对附件 svddata.txt 中的矩阵作 SVD 分解 
$A = P\Sigma Q$。并计算 $PP^T-I, QQ^T-I, P\Sigma Q-A$ 的绝对值最大的元素,依次用ep, eq, et 表示。

\textbf{要求输出迭代次数,从小到大排序的所有奇异值以及上面要求的三个值。}

输出格式为:(可以更详细,不能比下面的简单)

迭代次数:x

奇异值从小到大:

xxxxxxxxxxxxxxxxxxxxxxxxxxx

ep = xx

eq = xx

et = xx

以下内容不需要在报告中给出,但要在上交的程序中输出。

A=PTQ(可以用别的字母,但是要在最上面说明)

T=

[矩阵]

P=

[矩阵]

Q=

[矩阵]


\section*{二、作业涉及的算法}

必须实现的算法:
参考课本 7.6.2 节 (P234-240)SVD 迭代完成 SVD 算法 7.6.3

\textbf{务必先看懂算法的描述再写代码!!!}


\section*{三、附加说明}
1. 尽量使用 c++ 和 visual studio。

2. 本次作业最迟ddl 为 \textbf{2024.1.6(周六)23:59} ,请大家尽早提交。超时作业没有特殊情况者拒收。
\textbf{请注意,这也是所有作业的最后补交ddl!!!迟交作业得分*0.6。}

3. 请确保你的程序能顺利跑出正确的结果再上交!
可以用 Mathematica/Matlab 等工具来验证你的解是否正确。

4. 没有报告的程序作业不予批改,报告一定要交pdf版本。

\end{document}