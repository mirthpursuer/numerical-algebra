\documentclass{article}
\usepackage[UTF8]{ctex}
\usepackage{float,indentfirst,verbatim,fancyhdr,graphicx,listings,longtable,amsmath, amsfonts,amssymb}
\usepackage{algorithm}
\usepackage{amsmath}
\usepackage{algpseudocode}
\usepackage{float}
\usepackage{longtable}
\usepackage{lscape}
\textheight 23.5cm \textwidth 15.8cm
%\leftskip -1cm
\topmargin -1.5cm \oddsidemargin 0.3cm \evensidemargin -0.3cm
\usepackage[framemethod=TikZ]{mdframed}
\usepackage{url}   % 网页链接
\usepackage{subcaption} % 子标题
\usepackage[left=2.50cm, right=2.50cm, top=2.50cm, bottom=2.50cm]{geometry} %页边距
\usepackage{helvet}
\usepackage{amsmath, amsfonts, amssymb} % 数学公式、符号
%\usepackage[english]{babel}
\usepackage{graphicx}   % 图片
\usepackage{url}        % 超链接
\usepackage{bm}         % 加粗方程字体
\usepackage{multirow}
\usepackage{booktabs}
%\usepackage{algorithm}
%\usepackage{algorithmic}
\usepackage{esint}
\usepackage{hyperref} %bookmarks
\usepackage{fancyhdr} %设置页眉、页脚
%\hypersetup{colorlinks, bookmarks, unicode} %unicode
\usepackage{multicol}
\usepackage{graphicx}
\usepackage{xcolor}
\title{数值代数实验报告}
\author{PB21010483 郭忠炜}

\begin{document}
\maketitle

\section*{\centerline{一. 问题描述}}

\subsection*{Exercise1.1}

编写通用的QR分解算法子程序,将系数矩阵分解为正交矩阵Q和上三角矩阵R,然后编写子程序用于求解线性方程组。使用这些程序解决第一章上级习题中的三个方程组,比较各方法的结果,评估它们的优劣,输出计算结果、误差和运行时间。

\subsection*{Exercise1.2}

利用QR分解算法子程序,编写求解最小二乘线性问题的程序,以二次多项式形式 $y = a t^2 + b t + c$ 来拟合第二题的数据,以使残差向量的二范数最小化。需要输出拟合的参数 $a$、$b$、$c$,残差向量的二范数(表示拟合的好坏),以及程序的运行时间。

\subsection*{Exercise1.3}

使用线性模型 $y = x_0 + a_1x_1 + a_2x_2 + \ldots + a_{11}x_{11}$ 来拟合第三题的数据,并求出模型中参数 $x_0, x_1, \ldots, x_{11}$ 的最小二乘解。计算后输出参数的最小二乘结果,残差向量的二范数(表示拟合的好坏),以及程序的运行时间。

\section*{\centerline{二. 程序介绍}}

\subsection*{Exercise1.1}

\subsubsection*{计算向量的Householder变换:}
\begin{itemize}
	\item \textbf{函数描述:} \texttt{house} 函数用于对给定向量进行Householder变换。
	\item \textbf{使用方式:} 调用 \texttt{house(x, v, beta)} 函数,传入向量 $x$ 和向量 $d$ 与double型变量$beta$ 作为输出参数,函数会计算Householder变化并将结果存储在 $v$ 和 $beta$ 中。
\end{itemize}

\subsubsection*{Householder方法的QR分解:}
\begin{itemize}
	\item \textbf{函数描述:} \texttt{QRDecomposition} 函数用于对给定矩阵进行Householder方法的QR分解。
	\item \textbf{使用方式:} 调用 \texttt{QRDecomposition(A, d)} 函数,传入矩阵 $A$ 和一个空的向量 $d$ 作为输出参数,函数会计算QR分解并将结果存储在 $A$ 和 $d$ 中。
\end{itemize}

\subsubsection*{QR方程求解:}
\begin{itemize}
	\item \textbf{函数描述:} \texttt{QR\_equation\_solving} 函数用于通过QR分解求解线性方程组。
	\item \textbf{使用方式:} 调用 \texttt{QR\_equation\_solving(A, b)} 函数,传入系数矩阵 $A$ 和右侧向量 $b$,函数会使用QR分解计算方程组的解并返回结果。
\end{itemize}

\subsubsection*{QR分解的Householder矩阵生成:}
\begin{itemize}
	\item \textbf{函数描述:} \texttt{HouseholdMatrix} 函数用于生成Householder矩阵。
	\item \textbf{使用方式:} 调用 \texttt{HouseholdMatrix(A, d, k)} 函数,传入矩阵 $A$、向量 $d$ 和 $k$,函数会生成Householder矩阵并返回结果。
\end{itemize}

\subsection*{Exercise1.2 \& Exercise1.3}
\subsubsection*{最小二乘问题求解:}
\begin{itemize}
	\item \textbf{函数描述:} \texttt{LS\_proplem\_solving} 函数用于通过QR分解求解最小二乘问题。
	\item \textbf{使用方式:} 调用 \texttt{LS\_proplem\_solving(A, b)} 函数,传入系数矩阵 $A$ 和右侧向量 $b$,函数会使用QR分解计算最小二乘问题的解并返回结果。
\end{itemize}

\subsubsection*{向量的二范数 (Vector Two-Norm):}
\begin{itemize}
	\item \textbf{函数描述:} \texttt{VectorTwoNorm} 函数用于计算输入向量的二范数。
	\item \textbf{使用方式:} 调用 \texttt{VectorTwoNorm(x)} 函数,传入向量 $x$,函数会返回它的二范数。
\end{itemize}

\section*{\centerline{三. 实验结果}}

\subsection*{Exercise1.1}

\begin{table}[H]
	\centering
	\begin{tabular}{cccccccc}
		\hline
		\multirow{3}{*}{求解方程1} & 矩阵规模 & 10          & 30          & 50       & 55      & 56     & 84     \\ \cline{2-8} 
		& 计算误差 & 1.64313e-14 & 5.35813e-07 & 0.658149 & 30.5912 & inf    & inf    \\
		& 运行时间 & 0.004s      & 0.066s      & 0.413s   & 0.71s   & 0.997s & 2.825s \\ \hline
	\end{tabular}
	\caption{QR分解求解方程1}
	\label{tab:my-table}
	
	\begin{table}[H]
		\centering
		\begin{tabular}{@{}cccc@{}}
			\toprule
			求解84阶的方程1 & 不选主元        & 全主元         & 列主元         \\ \midrule
			计算误差      & 5.36838e+08 & 5.36838e+08 & 1.07374e+09 \\
			运行时间      & 0.067s      & 0.073s      & 0.092s      \\ \bottomrule
		\end{tabular}
		\caption{Gauss消去求解方程1}
		\label{tab:my-table}
	\end{table}
	
	\begin{table}[H]
		\centering
		\begin{tabular}{ccccc}
			\hline
			\multirow{3}{*}{求解方程2} & 矩阵规模 & 10          & 50          & 100         \\ \cline{2-5} 
			& 计算误差 & 3.50414e-16 & 4.44089e-16 & 5.55112e-16 \\
			& 运行时间 & 0.007s      & 0.255s      & 2.872s      \\ \hline
		\end{tabular}
		\caption{QR分解求解方程2}
		\label{tab:my-table}
	\end{table}
	
	\begin{table}[H]
		\centering
		\begin{tabular}{cccc}
			\hline
			求解100阶的方程2 & 不选主元        & 全主元      & 列主元         \\ \hline
			计算误差       & 2.22045e-16 & 0.808276 & 2.22045e-16 \\
			运行时间       & 0.030s      & 0.116s   & 0.103s      \\ \hline
		\end{tabular}
		\caption{Gauss消去法求解方程2}
		\label{tab:my-table}
	\end{table}
	
	\begin{table}[H]
		\centering
		\begin{tabular}{cccccl}
			\hline
			\multirow{3}{*}{求解方程3} & 矩阵规模 & 10          & 13      & 20      & 40      \\ \cline{2-6} 
			& 计算误差 & 0.000484919 & 9.15715 & 248.999 & 167.689 \\
			& 运行时间 & 0.004s      & 0.013s  & 0.013s  & 0.099s  \\ \hline
		\end{tabular}
		\caption{QR分解求解方程3}
		\label{tab:my-table}
	\end{table}
	
	\begin{table}[H]
		\centering
		\begin{tabular}{cccc}
			\hline
			求解40阶的方程3 & 不选主元    & 全主元     & 列主元     \\ \hline
			计算误差      & 115.617 & 929.253 & 115.617 \\
			运行时间      & 0.003s  & 0.007s  & 0.005s  \\ \hline
		\end{tabular}
		\caption{Gauss消去法求解方程3}
		\label{tab:my-table}
	\end{table}
	
\end{table}

\subsection*{Exercise1.2}

拟合多项式为:$y = t^2 + t + 1$, 残向量的二范数为:3.6545, 运行时间: 0.003 seconds.

\subsection*{Exercise1.3}

拟合得到的$x$为:

\begin{center}
	\begin{tabular}{|*{6}{c|}}
		\hline
		$x_0$ & $x_1$ & $x_2$ & $x_3$ & $x_4$ & $x_5$ \\
		%\hline
		2.07752 & 0.718888 & 9.6802 & 0.153506 & 13.6796 & 1.98683 \\
		\hline
		$x_6$ & $x_7$ & $x_8$ & $x_9$ & $x_{10}$ & $x_{11}$ \\%	\hline
		-0.958225 & -0.484023 & -0.0736469 & 1.0187 & 1.44352 & 2.90279 \\
		\hline
	\end{tabular}
\end{center}

房屋估价的拟合模型为:
\begin{equation*}
	\begin{split}
		y = 2.07752+0.718888a1 + 9.6802a2 + 0.153506a3 + 13.6796a4 + 1.98683a5 \\
		-0.958225a6  -0.484023a7  -0.0736469a8 + 1.0187a9 + 1.44352a10 + 2.90279a11
	\end{split}
\end{equation*}

残向量的二范数为:16.3404
运行时间: 0.071 seconds.

\section*{\centerline{四. 结果分析}}

\subsection*{Exercise1.1}

在用 QR分解求解线性方程组时,对于给定的方程组1:当矩阵规模不大于30时,算法的求解误差保持在相当良好的范围之内;当矩阵规模达到50时,与精确解(全1向量)的误差已经不容忽视;而当矩阵达到56阶乃至更大之后,QR分解求解线性方程组的误差会出现$inf$。

对比同规模矩阵的求解(实验结果中只展示了84阶方程1求解的计算误差和运行时间),Gauss消去表现出了一定的优越性。在矩阵规模较小时,Gauss消去与QR分解求解误差相差不大,运行时间为QR分解求解的两倍左右;当矩阵规模达到30时,QR分解求解的误差是Gauss消去的几百倍,而运行时间为Gauss消去的两倍多;当矩阵规模达到50时,Gauss消去的计算误差约为0.03左右,而QR分解求解约为0.66,不过二者的运行时间相近;但随着矩阵规模上涨到84,无论是QR分解求解还是Gauss消去,其解的误差都大到难以承受。对于方程2和方程3,Gauss消去也表现出了类似的优势。

回顾Gauss消去与QR分解,Gauss消去具有更低的计算复杂度,而QR分解的主要优势是它在处理病态矩阵时具有更好的数值稳定性,因为它采用了正交变换来减少舍入误差的传播,这些特性在本次实验中亦有体现。

总的来说,QR分解在处理小规模矩阵时具有较好的数值稳定性,随着矩阵规模的增加,计算复杂度急剧增加,可能导致数值不稳定性。相比之下,Gauss消去具有较低的计算复杂度,对于非病态矩阵可以在较大的规模下表现出较于QR分解的优势。

\subsection*{Exercise1.2 \& Exercise1.3}

对比两个问题的矩阵规模$(7*3$ 和 $28*12)$, 可以发现用QR分解编写线性最小二乘问题时,残向量二范数变化大致和矩阵规模变化成正比,而运行时间上随矩阵规模大致呈二次多项式变化趋势。

\end{document}