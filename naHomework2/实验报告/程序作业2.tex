\documentclass{article}
\usepackage[UTF8]{ctex}
\usepackage{float,indentfirst,verbatim,fancyhdr,graphicx,listings,longtable,amsmath, amsfonts,amssymb}
\usepackage{algorithm}
\usepackage{amsmath}
\usepackage{algpseudocode}
\usepackage{float}
\usepackage{longtable}
\usepackage{lscape}
\textheight 23.5cm \textwidth 15.8cm
%\leftskip -1cm
\topmargin -1.5cm \oddsidemargin 0.3cm \evensidemargin -0.3cm
\usepackage[framemethod=TikZ]{mdframed}
\usepackage{url}   % 网页链接
\usepackage{subcaption} % 子标题
\usepackage[left=2.50cm, right=2.50cm, top=2.50cm, bottom=2.50cm]{geometry} %页边距
\usepackage{helvet}
\usepackage{amsmath, amsfonts, amssymb} % 数学公式、符号
%\usepackage[english]{babel}
\usepackage{graphicx}   % 图片
\usepackage{url}        % 超链接
\usepackage{bm}         % 加粗方程字体
\usepackage{multirow}
\usepackage{booktabs}
%\usepackage{algorithm}
%\usepackage{algorithmic}
\usepackage{esint}
\usepackage{hyperref} %bookmarks
\usepackage{fancyhdr} %设置页眉、页脚
%\hypersetup{colorlinks, bookmarks, unicode} %unicode
\usepackage{multicol}
\usepackage{graphicx}
\usepackage{xcolor}
\title{数值代数实验报告}
\author{PB21010483 郭忠炜}

\begin{document}
\maketitle

\section*{\centerline{一、问题描述}}

\subsection*{Exercise1.1}

编写计算矩阵一范数的程序,估计 5 到 20 阶 Hilbert 矩阵的∞范数条件数。

\subsection*{Exercise1.2}

生成5到30阶的矩阵A,随机生成x,计算b=Ax,然后利用列主元高斯消去法解决Ax=b的线性方程组,最后估计解的精度并计算真实相对误差。

\section*{\centerline{二、程序介绍}}

\subsection*{Exercise1.1}

\subsubsection*{生成符号向量 (Sign Vector Generation):}
\begin{itemize}
	\item \textbf{函数描述:} \texttt{sign} 函数用于生成输入向量 $w$ 的符号向量,即返回一个与 $w$ 同样大小的向量,其中元素为 $w$ 中对应元素的符号。
	\item \textbf{使用方式:} 调用 \texttt{sign(w)} 函数,传入向量 $w$,函数返回符号向量。
\end{itemize}

\subsubsection*{计算向量内积 (Vector Inner Product):}
\begin{itemize}
	\item \textbf{函数描述:} \texttt{InnerProduct} 函数用于计算两个输入向量 $a$ 和 $b$ 的内积。
	\item \textbf{使用方式:} 调用 \texttt{InnerProduct(a, b)} 函数,传入两个向量 $a$ 和 $b$,函数返回它们的内积。
\end{itemize}

\subsubsection*{计算向量无穷范数 (Vector Infinity Norm):}
\begin{itemize}
	\item \textbf{函数描述:} \texttt{VectorInfinityNorm} 函数用于计算输入向量 $vec$ 的无穷范数,即返回 $vec$ 中绝对值最大的元素。
	\item \textbf{使用方式:} 调用 \texttt{VectorInfinityNorm(vec)} 函数,传入向量 $vec$,函数返回无穷范数。
\end{itemize}

\subsubsection*{计算向量一范数 (Vector One Norm):}
\begin{itemize}
	\item \textbf{函数描述:} \texttt{VectorOneNorm} 函数用于计算输入向量 $vec$ 的一范数,即返回 $vec$ 中所有元素的绝对值之和。
	\item \textbf{使用方式:} 调用 \texttt{VectorOneNorm(vec)} 函数,传入向量 $vec$,函数返回一范数。
\end{itemize}

\subsubsection*{生成对应下标为1单位向量 (Unit Vector Generation):}
\begin{itemize}
	\item \textbf{函数描述:} \texttt{UnitVectorGenerating} 函数用于生成一个与输入向量 $vec$ 同样大小的单位向量,其中单位向量的值对应于 $vec$ 的无穷范数下标。
	\item \textbf{使用方式:} 调用 \texttt{UnitVectorGenerating(vec, n)} 函数,传入向量 $vec$ 和整数 $n$,函数返回单位向量。
\end{itemize}

\subsubsection*{计算矩阵一范数 (Matrix One Norm):}
\begin{itemize}
	\item \textbf{函数描述:} \texttt{MatrixOneNorm} 函数用于计算输入矩阵 $A$ 的一范数,即返回 $A$ 中每列元素的绝对值之和的最大值。
	\item \textbf{使用方式:} 调用 \texttt{MatrixOneNorm(n, A)} 函数,传入整数 $n$ 和矩阵 $A$,函数返回一范数。
\end{itemize}

\subsubsection*{计算矩阵无穷范数 (Matrix Infinity Norm):}
\begin{itemize}
	\item \textbf{函数描述:} \texttt{MatrixInfinityNorm} 函数用于计算输入矩阵 $matrix$ 的无穷范数,即返回 $matrix$ 中每行元素的绝对值之和的最大值。
	\item \textbf{使用方式:} 调用 \texttt{MatrixInfinityNorm(matrix)} 函数,传入矩阵 $matrix$,函数返回无穷范数。
\end{itemize}

\subsection*{Exercise1.2}

\subsubsection*{矩阵向量乘法 (Matrix-Vector Multiplication):}
\begin{itemize}
	\item \textbf{函数描述:} \texttt{MatrixVectorMultiply} 函数用于计算矩阵 $A$ 和向量 $b$ 的乘积,返回一个向量。
	\item \textbf{使用方式:} 调用 \texttt{MatrixVectorMultiply(A, b)} 函数,传入矩阵 $A$ 和向量 $b$,函数返回乘积向量。
\end{itemize}

\subsubsection*{向量减法 (Vector Subtraction):}
\begin{itemize}
	\item \textbf{函数描述:} \texttt{VectorSubtraction} 函数用于计算两个输入向量 $x$ 和 $y$ 的差,返回一个向量。
	\item \textbf{使用方式:} 调用 \texttt{VectorSubtraction(x, y)} 函数,传入两个向量 $x$ 和 $y$,函数返回它们的差向量。
\end{itemize}

除了上面列举的函数之外,我还调用了第一章作业中定义的函数,比如生成Hilbert矩阵、矩阵转置和方程求解有关的程序。

\section*{\centerline{三、实验结果}}

\subsection*{Exercise1.1}


\begin{table}[htbp]
	\centering
	\begin{tabular}{cc}
		\toprule
		\textbf{矩阵规模} & \textbf{∞范数条件数} \\
		\midrule
		5             & 943656          \\
		6             & 2.90703e+07     \\
		7             & 9.85195e+08     \\
		8             & 3.38728e+10     \\
		9             & 1.09965e+12     \\
		10            & 3.53525e+13     \\
		11            & 1.22961e+15     \\
		12            & 3.82265e+16     \\
		13            & 5.50049e+17     \\
		14            & 3.19705e+18     \\
		15            & 1.02714e+18     \\
		16            & 6.27368e+18     \\
		17            & 3.80892e+18     \\
		18            & 4.3539e+18      \\
		19            & 4.43764e+18     \\
		20            & 4.45685e+18    \\
		\bottomrule
	\end{tabular}
	\caption{5 到 20 阶 Hilbert 矩阵的∞范数条件数}
	\label{tab:my-table}
\end{table}

\subsection*{Exercise1.2}


% Please add the following required packages to your document preamble:
% \usepackage{longtable}
% Note: It may be necessary to compile the document several times to get a multi-page table to line up properly
\begin{longtable}[c]{ccc}
	\toprule
	\textbf{矩阵规模} & \textbf{估计精度} & \textbf{真实精度} \\
	\endfirsthead
	%
	\endhead
	%
	\midrule
	5             & 1.39355e-15   & 5.37651e-16   \\
	6             & 3.99934e-16   & 2.03221e-16   \\
	7             & 4.44164e-16   & 2.05858e-16   \\
	8             & 3.4203e-15    & 1.41432e-15   \\
	9             & 3.35443e-16   & 2.25212e-16   \\
	10            & 1.14152e-14   & 4.78374e-15   \\
	11            & 1.39223e-14   & 5.84474e-15   \\
	12            & 1.3475e-13    & 6.0411e-14    \\
	13            & 7.64566e-14   & 3.19566e-14   \\
	14            & 7.04473e-14   & 3.33849e-14   \\
	15            & 1.34147e-12   & 6.02681e-13   \\
	\bottomrule
	\newpage
	\toprule
	16            & 8.65455e-14   & 3.94364e-14   \\
	17            & 7.7722e-12    & 2.86122e-12   \\
	18            & 1.837e-12     & 9.30363e-13   \\
	19            & 1.2035e-13    & 7.6454e-14    \\
	20            & 1.2426e-11    & 6.55229e-12   \\
	21            & 3.1232e-11    & 1.52572e-11   \\
	22            & 6.02992e-11   & 2.64836e-11   \\
	23            & 6.42282e-11   & 3.59628e-11   \\
	24            & 4.12898e-10   & 2.23124e-10   \\
	25            & 1.04553e-09   & 4.87306e-10   \\
	26            & 2.61624e-10   & 1.03894e-10   \\
	27            & 1.2612e-10    & 6.56837e-11   \\
	28            & 3.04428e-10   & 1.72571e-10   \\
	29            & 3.63452e-09   & 1.52792e-09   \\
	30            & 6.48284e-09   & 3.33615e-09  \\
	\bottomrule
	\caption{矩阵规模为5$\sim$30时的估算精度与真实精度}
	\label{tab:my-table}\\
\end{longtable}



\section*{\centerline{四、结果分析}}

\subsection*{Exercise1.1}

从运算结果来看,Hilbert∞范数条件数随着矩阵规模增大到14,其数量级迅速增大到1e+18,之后稳定在改数量级,这可能是由于Hilbert矩阵本身的性质。

\subsection*{Exercise1.2}

随着矩阵规模的增大,可以观察到估算精度和真实精度表现出了同步的增大,数量级上从$n=5$时的$1e-15$上涨到了$n=30$时$1e-9$, 但是估算精度与真实精度之间的比例几乎保持不变。


\end{document}