\documentclass{article}
\usepackage[UTF8]{ctex}
\usepackage{float,indentfirst,verbatim,fancyhdr,graphicx,listings,longtable,amsmath, amsfonts,amssymb}

\textheight 23.5cm \textwidth 15.8cm
%\leftskip -1cm
\topmargin -1.5cm \oddsidemargin 0.3cm \evensidemargin -0.3cm

%\pagestyle{fancy} \lhead{FDM Homework Template} \chead{}
%\rhead{\bfseries}
%
%\lfoot{} \cfoot{} \rfoot{\thepage}
%\renewcommand{\headrulewidth}{0.4pt}
%\renewcommand{\footrulewidth}{0.4pt}
%
\title{homework6}
\author{游瀚哲}

\begin{document}
\maketitle

\section*{一、作业要求}

1.求多项式方程的模最大根。

(1) 用 C++ 编制利用幂法求多项式方程 
$$f(x) = x^n + \alpha_{n-1}x^{n-1} +\cdots+ \alpha_1 x + \alpha_0 = 0 $$
的模最大根的通用子程序。

(2) 利用你所编制的子程序求下列各高次方程的模最大根。

(i) $x^3 + x^2 - 5x + 3 = 0;$

(ii) $x^3 - 3x - 1 = 0;$

(iii) $x^8 + 101x^7 + 208.01x^6 + 10891.01x^5 + 9802.08x^4 + 79108.9x^3-99902x^2 + 790x-1000 = 0.$

\textbf{要求输出迭代次数,用时和最大根的值(注意正负)}


2. 求实矩阵的全部特征值。

(1) 用 C++ 编制利用隐式 QR 算法 (课本算法 6.4.3) 求一个实矩阵的全部特征值的通用子程序。

(2) 利用你所编制的子程序计算方程 
$$x^{41} + x^3 + 1 = 0 $$

的全部根。

(3)设$A=\begin{bmatrix}
    \;9.1 & 3.0 & 2.6 & 4.0 \;\\
    \;4.2 & 5.3 & 4.7 & 1.6 \;\\
    \;3.2 & 1.7 & 9.4 & x \;\\
    \;6.1 & 4.9 & 3.5 & 6.2\; 
    \end{bmatrix}
        $

求当 x = 0.9, 1.0, 1.1 时 A 的全部特征值,并观察并在报告中叙述分析特征值实部、虚部和模长的变化情况。

\textbf{要求输出迭代次数、用时和所有特征值,复特征值用你认为合适的方式表示即可。}

\section*{二、作业涉及的算法}

必须实现的算法有:
幂法求模最大根参考课本 P165-166 的描述。
上 Hessenberg 分解参考课本 P181 算法 6.4.1。
双重步位移的 QR 迭代参考课本 P193 算法 6.4.2。
隐式 QR 算法参考课本 P194 算法 6.4.3。

\textbf{本次要实现的代码较多且较为复杂,务必先看懂算法的描述再写代码,注意老师说的实现细节}


\section*{三、附加说明}
1. 尽量使用 c++ 和 visual studio。

2. 本次作业最迟ddl 为 \textbf{2023.12.14(周四)23:59} ,请大家尽早提交。
超时作业没有特殊情况者拒收。若有特殊情况请提前私聊助教沟通。迟交的作业会视情况酌情扣分。

3. 请确保你的程序能顺利跑出正确的结果再上交!
可以用 Mathematica/Matlab 等工具来 验证你的解是否正确。

4. 没有报告的程序作业不予批改,报告一定要交pdf版本。

\end{document}