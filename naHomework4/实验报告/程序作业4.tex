\documentclass{article}
\usepackage[UTF8]{ctex}
\usepackage{float,indentfirst,verbatim,fancyhdr,graphicx,listings,longtable,amsmath, amsfonts,amssymb}
\usepackage{algorithm}
\usepackage{amsmath}
\usepackage{algpseudocode}
\usepackage{float}
\usepackage{longtable}
\usepackage{lscape}
\usepackage{colortbl}
\usepackage{array}
\usepackage{multirow}
\textheight 23.5cm \textwidth 15.8cm
%\leftskip -1cm
\topmargin -1.5cm \oddsidemargin 0.3cm \evensidemargin -0.3cm
\usepackage[framemethod=TikZ]{mdframed}
\usepackage{url}   % 网页链接
\usepackage{subcaption} % 子标题
\usepackage[left=2.50cm, right=2.50cm, top=2.50cm, bottom=2.50cm]{geometry} %页边距
\usepackage{helvet}
\usepackage{amsmath, amsfonts, amssymb} % 数学公式、符号
%\usepackage[english]{babel}
\usepackage{graphicx}   % 图片
\usepackage{url}        % 超链接
\usepackage{bm}         % 加粗方程字体
\usepackage{multirow}
\usepackage{booktabs}
%\usepackage{algorithm}
%\usepackage{algorithmic}
\usepackage{esint}
\usepackage{hyperref} %bookmarks
\usepackage{fancyhdr} %设置页眉、页脚
%\hypersetup{colorlinks, bookmarks, unicode} %unicode
\usepackage{multicol}
\usepackage{graphicx}
\usepackage{xcolor}
\title{数值代数实验报告}
\author{PB21010483 郭忠炜}

\begin{document}
\maketitle

\section*{\centerline{一. 问题描述}}

\subsection*{Exercise1}

对于给定的二阶微分方程边值问题,将问题离散化为线性方程组后,使用Jacobi迭代法、Gauss-Seidel迭代法和SOR迭代法求解该方程组。比较这些迭代方法在不同参数($\epsilon$)设置下的性能,并分析数值解与精确解之间的误差。

\subsection*{Exercise2}

对于二维偏微分方程,通过在 $[0,1]\times[0,1]$ 区域上进行均匀剖分,将方程离散化为一个代数方程组,采用中心差分方法得到差分方程。对于给定的函数 $g(x, y) = \exp(xy)$ 和 $f(x, y) = x + y$,使用Jacobi迭代法、Gauss-Seidel迭代法和SOR迭代法求解这个代数方程组。具体来说,我需要考虑不同的网格尺寸 $N=20,40,60$,并比较不同网格尺寸下求解过程中所需的迭代次数和相应的 CPU 时间。

根据讲义给出的推导思路,可以得到Jacobi迭代法、Gauss-Seidel迭代法和SOR迭代法的迭代格式如下。

Jacobi 迭代格式:
$$Dx^{(k+1)}=(L+U)x^{(k)}+b$$
$$(4 + h^2g(ih, jh))u_{i,j}^{(k+1)}=u_{i-1,j}^{(k)}+u_{i,j-1}^{(k)}+u_{i+1,j}^{(k)}+u_{i,j+1}^{(k)}+h^2f(ih, jh)$$

G-S 迭代格式:
$$Dx^{(k+1)}=Lx^{(k+1)}+Ux^{(k)}+b$$
$$(4 + h^2g(ih, jh))u_{i,j}^{(k+1)}=u_{i-1,j}^{(k+1)}+u_{i,j-1}^{(k+1)}+u_{i+1,j}^{(k)}+u_{i,j+1}^{(k)}+h^2f(ih, jh)$$

SOR 迭代格式:
$$Dx^{(k+1)}=\omega Lx^{(k+1)}+((1-\omega)D+\omega U)x^{(k)}+\omega b$$
$$(4 + h^2g(ih, jh))u_{i,j}^{(k+1)}=\omega u_{i-1,j}^{(k+1)}+\omega u_{i,j-1}^{(k+1)}+(1-\omega)(4 + h^2g(ih, jh))u_{i,j}^{(k)}+\omega (u_{i+1,j}^{(k)}+u_{i,j+1}^{(k)})+\omega h^2f(ih, jh)$$


\section*{\centerline{二. 程序介绍}}

\subsubsection*{Jacobi迭代法:}
\begin{itemize}
	\item \textbf{函数描述:} \texttt{Jacobi\_Iteration} 函数用于通过Jacobi迭代法求解线性方程组。
	\item \textbf{使用方式:} 调用 \texttt{Jacobi\_Iteration(A, b)} 函数,传入系数矩阵 $A$ 和右侧向量 $b$,函数会进行Jacobi迭代,返回线性方程组的解。
\end{itemize}

\subsubsection*{Gauss-Seidel迭代法:}
\begin{itemize}
	\item \textbf{函数描述:} \texttt{GS\_Iteration} 函数用于通过Gauss-Seidel迭代法求解线性方程组。
	\item \textbf{使用方式:} 调用 \texttt{GS\_Iteration(A, b)} 函数,传入系数矩阵 $A$ 和右侧向量 $b$,函数会进行Gauss-Seidel迭代,返回线性方程组的解。
\end{itemize}

\subsubsection*{SOR迭代法:}
\begin{itemize}
	\item \textbf{函数描述:} \texttt{SOR\_Iteration} 函数用于通过SOR迭代法求解线性方程组。
	\item \textbf{使用方式:} 调用 \texttt{SOR\_Iteration(A, b, omega)} 函数,传入系数矩阵 $A$、右侧向量 $b$ 和松弛因子 $\omega$,函数会进行SOR迭代,返回线性方程组的解。
\end{itemize}

\subsubsection*{SOR迭代法性能评估:}
\begin{itemize}
	\item \textbf{函数描述:} \texttt{SOR\_Performance} 函数用于评估SOR迭代法在给定参数下的性能。
	\item \textbf{使用方式:} 调用 \texttt{SOR\_Performance(A, b, omega)} 函数,传入系数矩阵 $A$、右侧向量 $b$ 和松弛因子 $\omega$,函数会返回SOR迭代的收敛迭代次数。
\end{itemize}

\subsubsection*{松弛因子搜索:}
\begin{itemize}
	\item \textbf{函数描述:} \texttt{BisearchOmega} 函数用于通过二分法搜索最佳松弛因子。
	\item \textbf{使用方式:} 调用 \texttt{BisearchOmega(A, b)} 函数,传入系数矩阵 $A$ 和右侧向量 $b$,函数会通过二分法搜索最佳松弛因子并返回结果。
\end{itemize}

\subsubsection*{迭代过程展示:}
\begin{itemize}
	\item \textbf{函数描述:} \texttt{Iterations} 函数用于执行差分方程的迭代求解过程。
	\item \textbf{使用方式:} 调用 \texttt{Iterations(epsilon)} 函数,传入参数 $\epsilon$,函数会执行Jacobi、Gauss-Seidel和SOR迭代法,并展示每种方法的解以及运行时间。
\end{itemize}

%\subsubsection*{矩阵相减:}
%\begin{itemize}
%	\item \textbf{函数描述:} \texttt{MatrixSubtraction} 函数用于计算两个矩阵的元素-wise 相减结果。
%	\item \textbf{使用方式:} 调用 \texttt{MatrixSubtraction(x, y)} 函数,传入两个矩阵 $x$ 和 $y$,函数会返回它们的相减结果。
%\end{itemize}

\subsubsection*{Jacobi迭代法(二维):}
\begin{itemize}
	\item \textbf{函数描述:} \texttt{Jacobi\_Iteration2} 函数用于通过Jacobi迭代法求解二维偏微分方程。
	\item \textbf{使用方式:} 调用 \texttt{Jacobi\_Iteration2(u)} 函数,传入初始解矩阵 $u$,函数会执行Jacobi迭代,求解二维偏微分方程。
\end{itemize}

\subsubsection*{Gauss-Seidel迭代法(二维):}
\begin{itemize}
	\item \textbf{函数描述:} \texttt{GS\_Iteration2} 函数用于通过Gauss-Seidel迭代法求解二维偏微分方程。
	\item \textbf{使用方式:} 调用 \texttt{GS\_Iteration2(u)} 函数,传入初始解矩阵 $u$,函数会执行Gauss-Seidel迭代,求解二维偏微分方程。
\end{itemize}

\subsubsection*{SOR迭代法(二维):}
\begin{itemize}
	\item \textbf{函数描述:} \texttt{SOR\_Iteration2} 函数用于通过SOR迭代法求解二维偏微分方程。
	\item \textbf{使用方式:} 调用 \texttt{SOR\_Iteration2(u, omega)} 函数,传入初始解矩阵 $u$ 和松弛因子 $\omega$,函数会执行SOR迭代,求解二维偏微分方程。
\end{itemize}

\subsubsection*{SOR迭代法性能评估(二维):}
\begin{itemize}
	\item \textbf{函数描述:} \texttt{SOR\_Performance2} 函数用于评估SOR迭代法在给定参数下的性能。
	\item \textbf{使用方式:} 调用 \texttt{SOR\_Performance2(u, omega)} 函数,传入初始解矩阵 $u$ 和松弛因子 $\omega$,函数会返回SOR迭代的收敛迭代次数。
\end{itemize}

\subsubsection*{松弛因子搜索(二维):}
\begin{itemize}
	\item \textbf{函数描述:} \texttt{BisearchOmega2} 函数用于通过二分法搜索最佳松弛因子。
	\item \textbf{使用方式:} 调用 \texttt{BisearchOmega2(u)} 函数,传入初始解矩阵 $u$,函数会通过二分法搜索最佳松弛因子并返回结果。
\end{itemize}

\subsubsection*{迭代过程展示(二维):}
\begin{itemize}
	\item \textbf{函数描述:} \texttt{Iterations2} 函数用于执行二维偏微分方程的迭代求解过程。
	\item \textbf{使用方式:} 调用 \texttt{Iterations2(n)} 函数,传入参数 $n$,函数会执行Jacobi、Gauss-Seidel和SOR迭代法,并展示每种方法的解以及运行时间。
\end{itemize}

\section*{\centerline{三. 实验结果}}
\subsection*{Exercise1}
表格中展示了在不同的$\epsilon$时的精确解和三种迭代方法(Jacobi迭代法、Gauss-Seidel迭代法和SOR迭代法)的计算结果、迭代次数和运行时间。
\newpage

\begin{table}[H]
	\centering
	\begin{tabular}{|*{11}{c}|}
		\hline
		\multicolumn{11}{|c|}{\textbf{$\epsilon = 1.0$ 时的精确解}}\\
		\hline
		0.0129 & 0.0257 & 0.0384 & 0.0510 & 0.0636 & 0.0761 & 0.0885 & 0.1008 & 0.1131 & 0.1253 & 0.1374 \\
		0.1494 & 0.1614 & 0.1733 & 0.1852 & 0.1970 & 0.2087 & 0.2203 & 0.2319 & 0.2434 & 0.2548 & 0.2662 \\
		0.2775 & 0.2888 & 0.3000 & 0.3111 & 0.3222 & 0.3332 & 0.3441 & 0.3550 & 0.3658 & 0.3766 & 0.3873 \\
		0.3980 & 0.4086 & 0.4191 & 0.4296 & 0.4401 & 0.4504 & 0.4608 & 0.4710 & 0.4813 & 0.4914 & 0.5016 \\
		0.5116 & 0.5217 & 0.5316 & 0.5415 & 0.5514 & 0.5612 & 0.5710 & 0.5807 & 0.5904 & 0.6000 & 0.6096 \\
		0.6192 & 0.6287 & 0.6381 & 0.6475 & 0.6569 & 0.6662 & 0.6755 & 0.6847 & 0.6939 & 0.7031 & 0.7122 \\
		0.7212 & 0.7303 & 0.7392 & 0.7482 & 0.7571 & 0.7660 & 0.7748 & 0.7836 & 0.7924 & 0.8011 & 0.8098 \\
		0.8184 & 0.8270 & 0.8356 & 0.8441 & 0.8526 & 0.8611 & 0.8695 & 0.8779 & 0.8863 & 0.8946 & 0.9029 \\
		0.9112 & 0.9194 & 0.9276 & 0.9358 & 0.9439 & 0.9520 & 0.9601 & 0.9681 & 0.9761 & 0.9841 & 0.9921 \\
		\hline
		\multicolumn{3}{|c}{\textbf{Jacobi迭代法}} & \multicolumn{4}{|c}{迭代次数: 13172}\ & \multicolumn{4}{|c|}{运行时间: 11.4540s}\\
		\hline
		0.0128 & 0.0256 & 0.0382 & 0.0508 & 0.0633 & 0.0758 & 0.0881 & 0.1004 & 0.1126 & 0.1248 & 0.1369 \\
		0.1489 & 0.1608 & 0.1727 & 0.1845 & 0.1962 & 0.2079 & 0.2195 & 0.2310 & 0.2425 & 0.2539 & 0.2653 \\
		0.2766 & 0.2878 & 0.2990 & 0.3101 & 0.3211 & 0.3321 & 0.3430 & 0.3539 & 0.3647 & 0.3755 & 0.3862 \\
		0.3968 & 0.4074 & 0.4179 & 0.4284 & 0.4388 & 0.4492 & 0.4595 & 0.4698 & 0.4800 & 0.4902 & 0.5003 \\
		0.5104 & 0.5204 & 0.5304 & 0.5403 & 0.5502 & 0.5600 & 0.5698 & 0.5795 & 0.5892 & 0.5988 & 0.6084 \\
		0.6180 & 0.6275 & 0.6370 & 0.6464 & 0.6558 & 0.6651 & 0.6744 & 0.6837 & 0.6929 & 0.7020 & 0.7112 \\
		0.7203 & 0.7293 & 0.7383 & 0.7473 & 0.7562 & 0.7651 & 0.7740 & 0.7828 & 0.7916 & 0.8003 & 0.8090 \\
		0.8177 & 0.8263 & 0.8349 & 0.8435 & 0.8520 & 0.8605 & 0.8690 & 0.8774 & 0.8858 & 0.8942 & 0.9025 \\
		0.9108 & 0.9191 & 0.9273 & 0.9355 & 0.9437 & 0.9518 & 0.9599 & 0.9680 & 0.9760 & 0.9841 & 0.9920 \\
		\hline
		
		\multicolumn{3}{|c}{\textbf{G-S迭代法}} & \multicolumn{4}{|c}{迭代次数: 6574}\ & \multicolumn{4}{|c|}{运行时间: 3.9850s}\\
		\hline
		0.0128 & 0.0256 & 0.0382 & 0.0508 & 0.0633 & 0.0757 & 0.0881 & 0.1004 & 0.1126 & 0.1248 & 0.1368 \\
		0.1488 & 0.1608 & 0.1727 & 0.1845 & 0.1962 & 0.2079 & 0.2195 & 0.2310 & 0.2425 & 0.2539 & 0.2652 \\
		0.2765 & 0.2878 & 0.2989 & 0.3100 & 0.3211 & 0.3321 & 0.3430 & 0.3539 & 0.3647 & 0.3754 & 0.3861 \\
		0.3968 & 0.4074 & 0.4179 & 0.4284 & 0.4388 & 0.4492 & 0.4595 & 0.4698 & 0.4800 & 0.4902 & 0.5003 \\
		0.5103 & 0.5204 & 0.5303 & 0.5403 & 0.5501 & 0.5600 & 0.5697 & 0.5795 & 0.5892 & 0.5988 & 0.6084 \\
		0.6180 & 0.6275 & 0.6369 & 0.6464 & 0.6557 & 0.6651 & 0.6744 & 0.6836 & 0.6928 & 0.7020 & 0.7111 \\
		0.7202 & 0.7293 & 0.7383 & 0.7473 & 0.7562 & 0.7651 & 0.7740 & 0.7828 & 0.7916 & 0.8003 & 0.8090 \\
		0.8177 & 0.8263 & 0.8349 & 0.8435 & 0.8520 & 0.8605 & 0.8690 & 0.8774 & 0.8858 & 0.8942 & 0.9025 \\
		0.9108 & 0.9191 & 0.9273 & 0.9355 & 0.9437 & 0.9518 & 0.9599 & 0.9680 & 0.9760 & 0.9841 & 0.9920 \\
		\hline
		\multicolumn{3}{|c}{\textbf{SOR迭代法}} & \multicolumn{4}{|c}{迭代次数: 261}\ & \multicolumn{4}{|c|}{运行时间: 0.1650s}\\
		\hline
		0.0129 & 0.0256 & 0.0383 & 0.0509 & 0.0635 & 0.0760 & 0.0884 & 0.1007 & 0.1129 & 0.1251 & 0.1372 \\
		0.1493 & 0.1612 & 0.1731 & 0.1850 & 0.1967 & 0.2084 & 0.2201 & 0.2316 & 0.2431 & 0.2546 & 0.2660 \\
		0.2773 & 0.2885 & 0.2997 & 0.3108 & 0.3219 & 0.3329 & 0.3438 & 0.3547 & 0.3655 & 0.3763 & 0.3870 \\
		0.3977 & 0.4083 & 0.4188 & 0.4293 & 0.4398 & 0.4501 & 0.4605 & 0.4707 & 0.4810 & 0.4911 & 0.5013 \\
		0.5113 & 0.5213 & 0.5313 & 0.5412 & 0.5511 & 0.5609 & 0.5707 & 0.5804 & 0.5901 & 0.5998 & 0.6093 \\
		0.6189 & 0.6284 & 0.6378 & 0.6472 & 0.6566 & 0.6659 & 0.6752 & 0.6845 & 0.6937 & 0.7028 & 0.7119 \\
		0.7210 & 0.7300 & 0.7390 & 0.7480 & 0.7569 & 0.7658 & 0.7746 & 0.7834 & 0.7922 & 0.8009 & 0.8096 \\
		0.8182 & 0.8268 & 0.8354 & 0.8440 & 0.8525 & 0.8609 & 0.8694 & 0.8778 & 0.8862 & 0.8945 & 0.9028 \\
		0.9111 & 0.9193 & 0.9275 & 0.9357 & 0.9438 & 0.9520 & 0.9600 & 0.9681 & 0.9761 & 0.9841 & 0.9921 \\
		\hline
	\end{tabular}
	%\caption{$\epsilon = 1.0$ 时三种迭代法的输出对比}
\end{table}

\begin{table}[H]
	\centering
	\begin{tabular}{|*{11}{c}|}
		\hline
		\multicolumn{11}{|c|}{\textbf{$\epsilon = 0.1$ 时的精确解}}\\
		\hline
		0.0526 & 0.1006 & 0.1446 & 0.1848 & 0.2217 & 0.2556 & 0.2867 & 0.3153 & 0.3417 & 0.3661 & 0.3886 \\
		0.4094 & 0.4288 & 0.4467 & 0.4635 & 0.4791 & 0.4937 & 0.5074 & 0.5202 & 0.5324 & 0.5438 & 0.5546 \\
		0.5649 & 0.5747 & 0.5840 & 0.5929 & 0.6014 & 0.6096 & 0.6175 & 0.6251 & 0.6325 & 0.6396 & 0.6466 \\
		0.6533 & 0.6599 & 0.6664 & 0.6727 & 0.6788 & 0.6849 & 0.6909 & 0.6967 & 0.7025 & 0.7082 & 0.7139 \\
		0.7195 & 0.7250 & 0.7305 & 0.7359 & 0.7413 & 0.7467 & 0.7520 & 0.7573 & 0.7625 & 0.7678 & 0.7730 \\
		0.7782 & 0.7833 & 0.7885 & 0.7937 & 0.7988 & 0.8039 & 0.8090 & 0.8141 & 0.8192 & 0.8243 & 0.8293 \\
		0.8344 & 0.8395 & 0.8445 & 0.8496 & 0.8546 & 0.8596 & 0.8647 & 0.8697 & 0.8747 & 0.8798 & 0.8848 \\
		0.8898 & 0.8948 & 0.8999 & 0.9049 & 0.9099 & 0.9149 & 0.9199 & 0.9249 & 0.9299 & 0.9349 & 0.9399 \\
		0.9450 & 0.9500 & 0.9550 & 0.9600 & 0.9650 & 0.9700 & 0.9750 & 0.9800 & 0.9850 & 0.9900 & 0.9950 \\
		\hline
		\multicolumn{3}{|c}{\textbf{Jacobi迭代法}} & \multicolumn{4}{|c}{迭代次数: 5926}\ & \multicolumn{4}{|c|}{运行时间: 3.4670s}\\
		\hline
		0.0504 & 0.0967 & 0.1393 & 0.1784 & 0.2144 & 0.2476 & 0.2782 & 0.3066 & 0.3327 & 0.3570 & 0.3795 \\
		0.4005 & 0.4199 & 0.4381 & 0.4551 & 0.4709 & 0.4858 & 0.4998 & 0.5130 & 0.5254 & 0.5372 & 0.5483 \\
		0.5589 & 0.5690 & 0.5786 & 0.5878 & 0.5966 & 0.6051 & 0.6133 & 0.6211 & 0.6288 & 0.6361 & 0.6433 \\
		0.6502 & 0.6570 & 0.6637 & 0.6701 & 0.6765 & 0.6827 & 0.6888 & 0.6948 & 0.7007 & 0.7066 & 0.7123 \\
		0.7180 & 0.7237 & 0.7292 & 0.7348 & 0.7402 & 0.7457 & 0.7511 & 0.7564 & 0.7617 & 0.7670 & 0.7723 \\
		0.7775 & 0.7828 & 0.7880 & 0.7932 & 0.7983 & 0.8035 & 0.8086 & 0.8137 & 0.8189 & 0.8240 & 0.8291 \\
		0.8341 & 0.8392 & 0.8443 & 0.8494 & 0.8544 & 0.8595 & 0.8645 & 0.8696 & 0.8746 & 0.8797 & 0.8847 \\
		0.8897 & 0.8948 & 0.8998 & 0.9048 & 0.9098 & 0.9148 & 0.9199 & 0.9249 & 0.9299 & 0.9349 & 0.9399 \\
		0.9449 & 0.9499 & 0.9549 & 0.9600 & 0.9650 & 0.9700 & 0.9750 & 0.9800 & 0.9850 & 0.9900 & 0.9950 \\
		\hline
		\multicolumn{3}{|c}{\textbf{G-S迭代法}} & \multicolumn{4}{|c}{迭代次数: 2981}\ & \multicolumn{4}{|c|}{运行时间: 1.6220s}\\
		\hline
		0.0504 & 0.0967 & 0.1392 & 0.1784 & 0.2144 & 0.2476 & 0.2782 & 0.3065 & 0.3327 & 0.3570 & 0.3795 \\
		0.4004 & 0.4199 & 0.4381 & 0.4550 & 0.4709 & 0.4858 & 0.4998 & 0.5130 & 0.5254 & 0.5372 & 0.5483 \\
		0.5589 & 0.5690 & 0.5786 & 0.5878 & 0.5966 & 0.6051 & 0.6133 & 0.6211 & 0.6287 & 0.6361 & 0.6433 \\
		0.6502 & 0.6570 & 0.6636 & 0.6701 & 0.6765 & 0.6827 & 0.6888 & 0.6948 & 0.7007 & 0.7066 & 0.7123 \\
		0.7180 & 0.7237 & 0.7292 & 0.7348 & 0.7402 & 0.7457 & 0.7511 & 0.7564 & 0.7617 & 0.7670 & 0.7723 \\
		0.7775 & 0.7828 & 0.7880 & 0.7932 & 0.7983 & 0.8035 & 0.8086 & 0.8137 & 0.8189 & 0.8240 & 0.8291 \\
		0.8341 & 0.8392 & 0.8443 & 0.8494 & 0.8544 & 0.8595 & 0.8645 & 0.8696 & 0.8746 & 0.8797 & 0.8847 \\
		0.8897 & 0.8948 & 0.8998 & 0.9048 & 0.9098 & 0.9148 & 0.9199 & 0.9249 & 0.9299 & 0.9349 & 0.9399 \\
		0.9449 & 0.9499 & 0.9549 & 0.9600 & 0.9650 & 0.9700 & 0.9750 & 0.9800 & 0.9850 & 0.9900 & 0.9950 \\
		\hline
		\multicolumn{3}{|c}{\textbf{SOR迭代法}} & \multicolumn{4}{|c}{迭代次数: 201}\ & \multicolumn{4}{|c|}{运行时间: 0.1480s}\\
		\hline
		0.0505 & 0.0968 & 0.1394 & 0.1785 & 0.2146 & 0.2478 & 0.2784 & 0.3068 & 0.3330 & 0.3573 & 0.3798 \\
		0.4007 & 0.4202 & 0.4384 & 0.4553 & 0.4712 & 0.4861 & 0.5001 & 0.5133 & 0.5257 & 0.5375 & 0.5486 \\
		0.5592 & 0.5693 & 0.5789 & 0.5881 & 0.5969 & 0.6054 & 0.6135 & 0.6214 & 0.6290 & 0.6364 & 0.6435 \\
		0.6505 & 0.6572 & 0.6639 & 0.6703 & 0.6767 & 0.6829 & 0.6890 & 0.6950 & 0.7009 & 0.7067 & 0.7125 \\
		0.7182 & 0.7238 & 0.7294 & 0.7349 & 0.7404 & 0.7458 & 0.7512 & 0.7565 & 0.7618 & 0.7671 & 0.7724 \\
		0.7776 & 0.7829 & 0.7880 & 0.7932 & 0.7984 & 0.8035 & 0.8087 & 0.8138 & 0.8189 & 0.8240 & 0.8291 \\
		0.8342 & 0.8393 & 0.8443 & 0.8494 & 0.8545 & 0.8595 & 0.8646 & 0.8696 & 0.8746 & 0.8797 & 0.8847 \\
		0.8897 & 0.8948 & 0.8998 & 0.9048 & 0.9098 & 0.9148 & 0.9199 & 0.9249 & 0.9299 & 0.9349 & 0.9399 \\
		0.9449 & 0.9499 & 0.9550 & 0.9600 & 0.9650 & 0.9700 & 0.9750 & 0.9800 & 0.9850 & 0.9900 & 0.9950 \\
		\hline
	\end{tabular}
	%\caption{$\epsilon = 0.1$ 时三种迭代法的输出对比}
\end{table}

\begin{table}[H]
	\centering
	\begin{tabular}{|*{11}{c}|}
		\hline
		\multicolumn{11}{|c|}{\textbf{$\epsilon = 0.01$ 时的精确解}}\\
		\hline
		0.3211 & 0.4423 & 0.4901 & 0.5108 & 0.5216 & 0.5288 & 0.5345 & 0.5398 & 0.5449 & 0.5500 & 0.5550 \\
		0.5600 & 0.5650 & 0.5700 & 0.5750 & 0.5800 & 0.5850 & 0.5900 & 0.5950 & 0.6000 & 0.6050 & 0.6100 \\
		0.6150 & 0.6200 & 0.6250 & 0.6300 & 0.6350 & 0.6400 & 0.6450 & 0.6500 & 0.6550 & 0.6600 & 0.6650 \\
		0.6700 & 0.6750 & 0.6800 & 0.6850 & 0.6900 & 0.6950 & 0.7000 & 0.7050 & 0.7100 & 0.7150 & 0.7200 \\
		0.7250 & 0.7300 & 0.7350 & 0.7400 & 0.7450 & 0.7500 & 0.7550 & 0.7600 & 0.7650 & 0.7700 & 0.7750 \\
		0.7800 & 0.7850 & 0.7900 & 0.7950 & 0.8000 & 0.8050 & 0.8100 & 0.8150 & 0.8200 & 0.8250 & 0.8300 \\
		0.8350 & 0.8400 & 0.8450 & 0.8500 & 0.8550 & 0.8600 & 0.8650 & 0.8700 & 0.8750 & 0.8800 & 0.8850 \\
		0.8900 & 0.8950 & 0.9000 & 0.9050 & 0.9100 & 0.9150 & 0.9200 & 0.9250 & 0.9300 & 0.9350 & 0.9400 \\
		0.9450 & 0.9500 & 0.9550 & 0.9600 & 0.9650 & 0.9700 & 0.9750 & 0.9800 & 0.9850 & 0.9900 & 0.9950 \\
		\hline
		\multicolumn{3}{|c}{\textbf{Jacobi迭代法}} & \multicolumn{4}{|c}{迭代次数: 569}\ & \multicolumn{4}{|c|}{运行时间: 0.3370s}\\
		\hline
		0.2550 & 0.3850 & 0.4525 & 0.4887 & 0.5094 & 0.5222 & 0.5311 & 0.5380 & 0.5440 & 0.5495 & 0.5548 \\
		0.5599 & 0.5649 & 0.5700 & 0.5750 & 0.5800 & 0.5850 & 0.5900 & 0.5950 & 0.6000 & 0.6050 & 0.6100 \\
		0.6150 & 0.6200 & 0.6250 & 0.6300 & 0.6350 & 0.6400 & 0.6450 & 0.6500 & 0.6550 & 0.6600 & 0.6650 \\
		0.6700 & 0.6750 & 0.6800 & 0.6850 & 0.6900 & 0.6950 & 0.7000 & 0.7050 & 0.7100 & 0.7150 & 0.7200 \\
		0.7250 & 0.7300 & 0.7350 & 0.7400 & 0.7450 & 0.7500 & 0.7550 & 0.7600 & 0.7650 & 0.7700 & 0.7750 \\
		0.7800 & 0.7850 & 0.7900 & 0.7950 & 0.8000 & 0.8050 & 0.8100 & 0.8150 & 0.8200 & 0.8250 & 0.8300 \\
		0.8350 & 0.8400 & 0.8450 & 0.8500 & 0.8550 & 0.8600 & 0.8650 & 0.8700 & 0.8750 & 0.8800 & 0.8850 \\
		0.8900 & 0.8950 & 0.9000 & 0.9050 & 0.9100 & 0.9150 & 0.9200 & 0.9250 & 0.9300 & 0.9350 & 0.9400 \\
		0.9450 & 0.9500 & 0.9550 & 0.9600 & 0.9650 & 0.9700 & 0.9750 & 0.9800 & 0.9850 & 0.9900 & 0.9950 \\
		\hline
		\multicolumn{3}{|c}{\textbf{G-S迭代法}} & \multicolumn{4}{|c}{迭代次数: 333}\ & \multicolumn{4}{|c|}{运行时间: 0.2010s}\\
		\hline
		0.2550 & 0.3850 & 0.4525 & 0.4887 & 0.5094 & 0.5222 & 0.5311 & 0.5380 & 0.5440 & 0.5495 & 0.5548 \\
		0.5599 & 0.5649 & 0.5700 & 0.5750 & 0.5800 & 0.5850 & 0.5900 & 0.5950 & 0.6000 & 0.6050 & 0.6100 \\
		0.6150 & 0.6200 & 0.6250 & 0.6300 & 0.6350 & 0.6400 & 0.6450 & 0.6500 & 0.6550 & 0.6600 & 0.6650 \\
		0.6700 & 0.6750 & 0.6800 & 0.6850 & 0.6900 & 0.6950 & 0.7000 & 0.7050 & 0.7100 & 0.7150 & 0.7200 \\
		0.7250 & 0.7300 & 0.7350 & 0.7400 & 0.7450 & 0.7500 & 0.7550 & 0.7600 & 0.7650 & 0.7700 & 0.7750 \\
		0.7800 & 0.7850 & 0.7900 & 0.7950 & 0.8000 & 0.8050 & 0.8100 & 0.8150 & 0.8200 & 0.8250 & 0.8300 \\
		0.8350 & 0.8400 & 0.8450 & 0.8500 & 0.8550 & 0.8600 & 0.8650 & 0.8700 & 0.8750 & 0.8800 & 0.8850 \\
		0.8900 & 0.8950 & 0.9000 & 0.9050 & 0.9100 & 0.9150 & 0.9200 & 0.9250 & 0.9300 & 0.9350 & 0.9400 \\
		0.9450 & 0.9500 & 0.9550 & 0.9600 & 0.9650 & 0.9700 & 0.9750 & 0.9800 & 0.9850 & 0.9900 & 0.9950 \\
		\hline
		\multicolumn{3}{|c}{\textbf{SOR迭代法}} & \multicolumn{4}{|c}{迭代次数: 101}\ & \multicolumn{4}{|c|}{运行时间: 0.0820s}\\
		\hline
		0.2550 & 0.3850 & 0.4525 & 0.4888 & 0.5094 & 0.5222 & 0.5311 & 0.5380 & 0.5440 & 0.5495 & 0.5548 \\
		0.5599 & 0.5649 & 0.5700 & 0.5750 & 0.5800 & 0.5850 & 0.5900 & 0.5950 & 0.6000 & 0.6050 & 0.6100 \\
		0.6150 & 0.6200 & 0.6250 & 0.6300 & 0.6350 & 0.6400 & 0.6450 & 0.6500 & 0.6550 & 0.6600 & 0.6650 \\
		0.6700 & 0.6750 & 0.6800 & 0.6850 & 0.6900 & 0.6950 & 0.7000 & 0.7050 & 0.7100 & 0.7150 & 0.7200 \\
		0.7250 & 0.7300 & 0.7350 & 0.7400 & 0.7450 & 0.7500 & 0.7550 & 0.7600 & 0.7650 & 0.7700 & 0.7750 \\
		0.7800 & 0.7850 & 0.7900 & 0.7950 & 0.8000 & 0.8050 & 0.8100 & 0.8150 & 0.8200 & 0.8250 & 0.8300 \\
		0.8350 & 0.8400 & 0.8450 & 0.8500 & 0.8550 & 0.8600 & 0.8650 & 0.8700 & 0.8750 & 0.8800 & 0.8850 \\
		0.8900 & 0.8950 & 0.9000 & 0.9050 & 0.9100 & 0.9150 & 0.9200 & 0.9250 & 0.9300 & 0.9350 & 0.9400 \\
		0.9450 & 0.9500 & 0.9550 & 0.9600 & 0.9650 & 0.9700 & 0.9750 & 0.9800 & 0.9850 & 0.9900 & 0.9950 \\
		\hline
	\end{tabular}
	%\caption{$\epsilon = 0.0100$ 时三种迭代法的输出对比}
\end{table}

\begin{table}[H]
	\centering
	\begin{tabular}{|*{11}{c}|}
		\hline
		\multicolumn{11}{|c|}{\textbf{$\epsilon = 0.0001$ 时的精确解}}\\
		\hline
		0.5050 & 0.5100 & 0.5150 & 0.5200 & 0.5250 & 0.5300 & 0.5350 & 0.5400 & 0.5450 & 0.5500 & 0.5550 \\
		0.5600 & 0.5650 & 0.5700 & 0.5750 & 0.5800 & 0.5850 & 0.5900 & 0.5950 & 0.6000 & 0.6050 & 0.6100 \\
		0.6150 & 0.6200 & 0.6250 & 0.6300 & 0.6350 & 0.6400 & 0.6450 & 0.6500 & 0.6550 & 0.6600 & 0.6650 \\
		0.6700 & 0.6750 & 0.6800 & 0.6850 & 0.6900 & 0.6950 & 0.7000 & 0.7050 & 0.7100 & 0.7150 & 0.7200 \\
		0.7250 & 0.7300 & 0.7350 & 0.7400 & 0.7450 & 0.7500 & 0.7550 & 0.7600 & 0.7650 & 0.7700 & 0.7750 \\
		0.7800 & 0.7850 & 0.7900 & 0.7950 & 0.8000 & 0.8050 & 0.8100 & 0.8150 & 0.8200 & 0.8250 & 0.8300 \\
		0.8350 & 0.8400 & 0.8450 & 0.8500 & 0.8550 & 0.8600 & 0.8650 & 0.8700 & 0.8750 & 0.8800 & 0.8850 \\
		0.8900 & 0.8950 & 0.9000 & 0.9050 & 0.9100 & 0.9150 & 0.9200 & 0.9250 & 0.9300 & 0.9350 & 0.9400 \\
		0.9450 & 0.9500 & 0.9550 & 0.9600 & 0.9650 & 0.9700 & 0.9750 & 0.9800 & 0.9850 & 0.9900 & 0.9950 \\
		\hline
		\multicolumn{3}{|c}{\textbf{Jacobi迭代法}} & \multicolumn{4}{|c}{迭代次数: 118}\ & \multicolumn{4}{|c|}{运行时间: 0.0840s}\\
		\hline
		0.5000 & 0.5100 & 0.5150 & 0.5200 & 0.5250 & 0.5300 & 0.5350 & 0.5400 & 0.5450 & 0.5500 & 0.5550 \\
		0.5600 & 0.5650 & 0.5700 & 0.5750 & 0.5800 & 0.5850 & 0.5900 & 0.5950 & 0.6000 & 0.6050 & 0.6100 \\
		0.6150 & 0.6200 & 0.6250 & 0.6300 & 0.6350 & 0.6400 & 0.6450 & 0.6500 & 0.6550 & 0.6600 & 0.6650 \\
		0.6700 & 0.6750 & 0.6800 & 0.6850 & 0.6900 & 0.6950 & 0.7000 & 0.7050 & 0.7100 & 0.7150 & 0.7200 \\
		0.7250 & 0.7300 & 0.7350 & 0.7400 & 0.7450 & 0.7500 & 0.7550 & 0.7600 & 0.7650 & 0.7700 & 0.7750 \\
		0.7800 & 0.7850 & 0.7900 & 0.7950 & 0.8000 & 0.8050 & 0.8100 & 0.8150 & 0.8200 & 0.8250 & 0.8300 \\
		0.8350 & 0.8400 & 0.8450 & 0.8500 & 0.8550 & 0.8600 & 0.8650 & 0.8700 & 0.8750 & 0.8800 & 0.8850 \\
		0.8900 & 0.8950 & 0.9000 & 0.9050 & 0.9100 & 0.9150 & 0.9200 & 0.9250 & 0.9300 & 0.9350 & 0.9400 \\
		0.9450 & 0.9500 & 0.9550 & 0.9600 & 0.9650 & 0.9700 & 0.9750 & 0.9800 & 0.9850 & 0.9900 & 0.9950 \\
		\hline
		\multicolumn{3}{|c}{\textbf{G-S迭代法}} & \multicolumn{4}{|c}{迭代次数: 109}\ & \multicolumn{4}{|c|}{运行时间: 0.0820s}\\
		\hline
		0.5000 & 0.5100 & 0.5150 & 0.5200 & 0.5250 & 0.5300 & 0.5350 & 0.5400 & 0.5450 & 0.5500 & 0.5550 \\
		0.5600 & 0.5650 & 0.5700 & 0.5750 & 0.5800 & 0.5850 & 0.5900 & 0.5950 & 0.6000 & 0.6050 & 0.6100 \\
		0.6150 & 0.6200 & 0.6250 & 0.6300 & 0.6350 & 0.6400 & 0.6450 & 0.6500 & 0.6550 & 0.6600 & 0.6650 \\
		0.6700 & 0.6750 & 0.6800 & 0.6850 & 0.6900 & 0.6950 & 0.7000 & 0.7050 & 0.7100 & 0.7150 & 0.7200 \\
		0.7250 & 0.7300 & 0.7350 & 0.7400 & 0.7450 & 0.7500 & 0.7550 & 0.7600 & 0.7650 & 0.7700 & 0.7750 \\
		0.7800 & 0.7850 & 0.7900 & 0.7950 & 0.8000 & 0.8050 & 0.8100 & 0.8150 & 0.8200 & 0.8250 & 0.8300 \\
		0.8350 & 0.8400 & 0.8450 & 0.8500 & 0.8550 & 0.8600 & 0.8650 & 0.8700 & 0.8750 & 0.8800 & 0.8850 \\
		0.8900 & 0.8950 & 0.9000 & 0.9050 & 0.9100 & 0.9150 & 0.9200 & 0.9250 & 0.9300 & 0.9350 & 0.9400 \\
		0.9450 & 0.9500 & 0.9550 & 0.9600 & 0.9650 & 0.9700 & 0.9750 & 0.9800 & 0.9850 & 0.9900 & 0.9950 \\
		\hline
		\multicolumn{3}{|c}{\textbf{SOR迭代法}} & \multicolumn{4}{|c}{迭代次数: 105}\ & \multicolumn{4}{|c|}{运行时间: 0.0860s}\\
		\hline
		0.5000 & 0.5100 & 0.5150 & 0.5200 & 0.5250 & 0.5300 & 0.5350 & 0.5400 & 0.5450 & 0.5500 & 0.5550 \\
		0.5600 & 0.5650 & 0.5700 & 0.5750 & 0.5800 & 0.5850 & 0.5900 & 0.5950 & 0.6000 & 0.6050 & 0.6100 \\
		0.6150 & 0.6200 & 0.6250 & 0.6300 & 0.6350 & 0.6400 & 0.6450 & 0.6500 & 0.6550 & 0.6600 & 0.6650 \\
		0.6700 & 0.6750 & 0.6800 & 0.6850 & 0.6900 & 0.6950 & 0.7000 & 0.7050 & 0.7100 & 0.7150 & 0.7200 \\
		0.7250 & 0.7300 & 0.7350 & 0.7400 & 0.7450 & 0.7500 & 0.7550 & 0.7600 & 0.7650 & 0.7700 & 0.7750 \\
		0.7800 & 0.7850 & 0.7900 & 0.7950 & 0.8000 & 0.8050 & 0.8100 & 0.8150 & 0.8200 & 0.8250 & 0.8300 \\
		0.8350 & 0.8400 & 0.8450 & 0.8500 & 0.8550 & 0.8600 & 0.8650 & 0.8700 & 0.8750 & 0.8800 & 0.8850 \\
		0.8900 & 0.8950 & 0.9000 & 0.9050 & 0.9100 & 0.9150 & 0.9200 & 0.9250 & 0.9300 & 0.9350 & 0.9400 \\
		0.9450 & 0.9500 & 0.9550 & 0.9600 & 0.9650 & 0.9700 & 0.9750 & 0.9800 & 0.9850 & 0.9900 & 0.9950 \\
		\hline
	\end{tabular}
	%\caption{$\epsilon = 0.0001$ 时三种迭代法的输出对比}
\end{table}

\subsection*{Exercise2}

\begin{table}[H]
	\centering
	\begin{tabular}{|c|c|c|c|c|}
		\hline
		\textbf{N}          & \textbf{迭代方法}     & \textbf{最小分量} & \textbf{迭代次数}           & \textbf{运行时间/s} \\ \hline
		\multirow{3}{*}{20} & Jacobi            & \multirow{9}{*}{0.9187} & 1122 & 0.4540       \\ \cline{2-2} \cline{4-5} 
		& G-S               &                         & 589 & 0.2540        \\ \cline{2-2} \cline{4-5} 
		& SOR($\omega$=1.7266) &                         & 68  & 0.0360        \\ \cline{1-2} \cline{4-5} 
		\multirow{3}{*}{40} & Jacobi            &                         & 4289 & 5.5790        \\ \cline{2-2} \cline{4-5} 
		& G-S               &                         & 2251 & 2.2560        \\ \cline{2-2} \cline{4-5} 
		& SOR($\omega$=1.8516) &                         & 134  & 1.550         \\ \cline{1-2} \cline{4-5} 
		\multirow{3}{*}{60} & Jacobi            &                         & 9376 & 26.6210       \\ \cline{2-2} \cline{4-5} 
		& G-S               &                         & 4927 & 11.3620       \\ \cline{2-2} \cline{4-5} 
		& SOR($\omega$=1.8984) &                         & 200  & 0.4140        \\ \hline
	\end{tabular}
\end{table}
\section*{\centerline{四. 结果分析}}

\subsection*{Exercise1}

对于固定的$\epsilon = 1.0$,Jacobi迭代法、Gauss-Seidel迭代法和SOR迭代法表现出了性能上的显著提升。在迭代结果相差不大的情况下,G-S迭代法较于Jacobi迭代法实现了运行时间的减半,而SOR迭代法的运行速度在经过选取最佳的参数$\omega$之后可以达到G-S迭代法的10倍有余。随着$\epsilon$不断减小,迭代矩阵$A$的性质发生变化,这体现在三大迭代法的迭代次数和运行时间渐渐趋近。

注:SOR迭代法中选取参数$\omega$的方法是在$[1,2]$区间内进行二分查找,以迭代次数为评价指标,在代码中对应的函数是\texttt{SOR\_Performance}和\texttt{SOR\_Performance}。

\subsection*{Exercise2}

从计算结果上看,对于固定的$N = 20$,Jacobi迭代法、Gauss-Seidel迭代法和SOR迭代法都完成了计算任务,但从计算性能上三者表现和Exercise1类似,并且随着划分细度从20增加到60后,SOR迭代法的优势越发突出。

\end{document}