\documentclass{article}
\usepackage[UTF8]{ctex}
\usepackage{float,indentfirst,verbatim,fancyhdr,graphicx,listings,longtable,amsmath, amsfonts,amssymb}

\textheight 23.5cm \textwidth 15.8cm
%\leftskip -1cm
\topmargin -1.5cm \oddsidemargin 0.3cm \evensidemargin -0.3cm

%\pagestyle{fancy} \lhead{FDM Homework Template} \chead{}
%\rhead{\bfseries}
%
%\lfoot{} \cfoot{} \rfoot{\thepage}
%\renewcommand{\headrulewidth}{0.4pt}
%\renewcommand{\footrulewidth}{0.4pt}
%
\title{homework7}
\author{游瀚哲}

\begin{document}
\maketitle

\section*{一、作业要求}
1.求实对称三对角阵的全部特征值和特征向量。

(1) 用 C++ 编制利用过关 Jacobi 方法求实对称三对角阵全部特征值和特征向量的通用子程序。

(2) 利用你所编制的子程序求50, 60, 70, 80, 90, 100 阶矩阵
$$
A=\begin{bmatrix}
\;4 & 1 & 0 & 0 & \cdots & 0 \;\\
\;1 & 4 & 1 & 0 & \cdots & 0 \;\\
\;0 & 1 & 4 & 1 & \cdots & 0 \;\\
\;\vdots & & \ddots & \ddots & \ddots \;\\
\;0 & \cdots & 0 & 1 & 4 & 1\;\\
\;0 & 0 & \cdots & 0 & 1 & 4\; 
\end{bmatrix}
$$
的全部特征值和特征向量。

参考课本 P217,设你的程序经过 k 步迭代后停止了,得到 $Q_k = J_1 \cdots J_k$ 及 $AQ_k = Q_kA_k$,
\textbf{要求程序直接输出 50, 60, 70, 80, 90, 100 阶的 $Q_k$ 和 $A_k$(指助教跑的时候可以直接看到所有结果)且给出每次求解的迭代次数和所用时间,
报告里只需给出 50 阶的两个矩阵截图,对于 60, 70, 80, 90, 100阶,请在报告中给出从小到大排序后的全部特征值。}

交上来的源文件中的参考输出格式:

n=xx, 迭代次数:x, 用时 xxx s.

Ak=

[矩阵]

Qk=

[矩阵]



2. 求实对称三对角阵的指定特征值及对应的特征向量.

(1) 用 C++ 编制先利用二分法求实对称三对角阵指定特征值,再利用反幂法求对应特征向量的通用子程序。

(2) 利用你所编制的子程序求100阶矩阵

$$
A=\begin{bmatrix}
\;2 & -1 & 0 & 0 & \cdots & 0 \;\\
\;-1 & 2 & -1 & 0 & \cdots & 0 \;\\
\;0 & -1 & 2 & -1 & \cdots & 0 \;\\
\;\vdots & & \ddots & \ddots & \ddots \;\\
\;0 & \cdots & 0 & -1 & 2 & -1\;\\
\;0 & 0 & \cdots & 0 & -1 & 2\; 
\end{bmatrix}
$$
的最大和最小特征值及对应的特征向量

\textbf{要求输出迭代次数,用时,特征值和特征向量。}

参考输出格式:最小特征值:x, 迭代次数:y,用时:z ms.

\section*{二、作业涉及的算法}

过关 Jacobi方法: 先看懂 P211-213 的经典 Jacobi 方法,再参考 P217 过关 Jacobi 方法的描述。
二分法:参考 P223 推论 7.4.1 下面的文字描述。其中变号数的计算见算法 7.4.1。
反幂法:P169 6.3 节开头的迭代格式。

\textbf{务必先看懂算法的描述再写代码!}


\section*{三、附加说明}
1. 尽量使用 c++ 和 visual studio。

2. 本次作业最迟ddl 为 \textbf{2023.12.28(周四)23:59} ,请大家尽早提交。
超时作业没有特殊情况者拒收。若有特殊情况请提前私聊助教沟通。迟交的作业会视情况酌情扣分。

3. 请确保你的程序能顺利跑出正确的结果再上交!
可以用 Mathematica/Matlab 等工具来 验证你的解是否正确。

4. 没有报告的程序作业不予批改,报告一定要交pdf版本。

\end{document}