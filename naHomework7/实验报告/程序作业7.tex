\documentclass{article}
\usepackage[UTF8]{ctex}
\usepackage{float,indentfirst,verbatim,fancyhdr,graphicx,listings,longtable,amsmath, amsfonts,amssymb}
\usepackage{algorithm}
\usepackage{amsmath}
\usepackage{algpseudocode}
\usepackage{float}
\usepackage{longtable}
\usepackage{lscape}
\usepackage{colortbl}
\usepackage{array}
\usepackage{multirow}
\textheight 23.5cm \textwidth 15.8cm
%\leftskip -1cm
\topmargin -1.5cm \oddsidemargin 0.3cm \evensidemargin -0.3cm
\usepackage[framemethod=TikZ]{mdframed}
\usepackage{url}   % 网页链接
\usepackage{subcaption} % 子标题
%\usepackage[a3paper, margin=2cm]{geometry}
\usepackage[left=2.50cm, right=2.50cm, top=2.50cm, bottom=2.50cm]{geometry} %页边距
\usepackage{helvet}
\usepackage{amsmath, amsfonts, amssymb} % 数学公式、符号
%\usepackage[english]{babel}
\usepackage{graphicx}   % 图片
\usepackage{url}        % 超链接
\usepackage{bm}         % 加粗方程字体
\usepackage{multirow}
\usepackage{booktabs}
%\usepackage{algorithm}
%\usepackage{algorithmic}
\usepackage{esint}
\usepackage{hyperref} %bookmarks
\usepackage{fancyhdr} %设置页眉、页脚
%\hypersetup{colorlinks, bookmarks, unicode} %unicode
\usepackage{multicol}
\usepackage{graphicx}
\usepackage{xcolor}
\title{数值代数实验报告}
\author{PB21010483 郭忠炜}

\begin{document}
\maketitle

\section*{\centerline{一. 问题描述}}

\subsection*{Exercise1}

编写一个 C++ 程序,利用 过关Jacobi 方法求解实对称三对角阵的全部特征值和特征向量。程序需要能处理 50、60、70、80、90 和 100 阶的矩阵,并在每次迭代后输出迭代次数、所用时间以及求解得到的 $Q_k$ 和 $A_k$。对于 50 阶的矩阵,需要提供截图显示两个矩阵的内容;对于 60、70、80、90 和 100 阶的矩阵,需要给出从小到大排序后的全部特征值。

\subsection*{Exercise2}

实现对实对称三对角阵进行特定特征值的求解,以及对应特征值的特征向量的计算。首先利用二分法求解所给矩阵的最大和最小特征值,接着使用反幂法来计算对应的特征向量。对于给定100阶三对角阵,要求输出最大和最小特征值及对应的特征向量与迭代次数、用时。

%\newpage
\section*{\centerline{二. 程序介绍}}

\subsubsection*{经典Jacobi方法}

\textbf{函数描述:} \texttt{jacobiClassic} 函数用于对矩阵 $A$ 的第 $p/q$ 行/列进行 Jacobi 方法操作,返回 Jacobi 旋转矩阵 $J$。

\textbf{使用方式:} 调用 \texttt{jacobiClassic(A, p, q)} 函数,传入矩阵 $A$ 及其需要进行 Jacobi 方法的行数 $p$ 和列数 $q$。函数将返回 Jacobi 旋转矩阵 $J$。

该函数实现了 Jacobi 方法的过程。首先确定 Jacobi 旋转矩阵,然后根据该矩阵对输入矩阵进行相应变换,并返回 Jacobi 旋转矩阵 $J$。

以下是函数的 C++ 代码实现,用于实现 Jacobi 方法:

\subsubsection*{矩阵非对角元素范数}

\textbf{函数描述:} \texttt{offDiagNorm} 函数用于计算矩阵 $A$ 的非对角元素范数。

\textbf{使用方式:} 调用 \texttt{offDiagNorm(A)} 函数,传入矩阵 $A$。函数将返回矩阵的非对角元素范数。

该函数通过迭代计算矩阵非对角元素的平方和,最终返回其平方根作为非对角元素范数。

\subsubsection*{判断矩阵是否达到阈值}

\textbf{函数描述:} \texttt{passingThreshold} 函数用于判断矩阵 $A$ 中的元素是否超过给定的阈值 $\delta$。

\textbf{使用方式:} 调用 \texttt{passingThreshold(A, delta)} 函数,传入矩阵 $A$ 和阈值 $\delta$。函数将返回一个布尔值,表示是否有元素超过了阈值。

该函数遍历矩阵 $A$ 中的元素,并与阈值 $\delta$ 进行比较,判断是否有元素超过了设定的阈值。

\subsubsection*{过关Jacobi 方法}

\textbf{函数描述:} \texttt{thresholdJacobi} 函数用于执行过关 Jacobi 方法,对矩阵 $A$ 进行迭代操作。

\textbf{使用方式:} 调用 \texttt{thresholdJacobi(A)} 函数,传入矩阵 $A$。函数将使用过关 Jacobi 方法进行迭代,并返回结果矩阵。

该函数利用 Jacobi 方法迭代操作矩阵 $A$,直至矩阵的非对角元素范数满足某个阈值 $\delta$ 或达到迭代次数上限。

\subsubsection*{计算变号数}

\textbf{函数描述:} \texttt{reversals} 函数用于计算矩阵 $A$ 的特定变号数。

\textbf{使用方式:} 调用 \texttt{reversals(A, miu)} 函数,传入矩阵 $A$ 和特定值 $\mu$。函数将返回矩阵的变号数。

该函数实现了计算矩阵 $A$ 的变号数的过程,根据特定的算法进行数值计算。

\subsubsection*{二分法求特征值}

\textbf{函数描述:} \texttt{dichEigenvalue} 函数用于通过二分法求解矩阵 $A$ 的第 $m$ 个特征值。

\textbf{使用方式:} 调用 \texttt{dichEigenvalue(A, m)} 函数,传入矩阵 $A$ 和特征值位置 $m$。函数将返回矩阵的第 $m$ 个特征值。

该函数利用二分法求解矩阵 $A$ 的特征值,根据特定的算法寻找满足条件的特征值。

\subsubsection*{向量相加}

\textbf{函数描述:} \texttt{VectorAddition} 函数用于执行两个向量的加法运算。

\textbf{使用方式:} 调用 \texttt{VectorAddition(x, y)} 函数,传入两个待相加的向量 $x$ 和 $y$。函数将返回相加后的向量。

该函数实现了两个向量的逐元素相加运算。

\subsubsection*{反幂法求特征向量}

\textbf{函数描述:} \texttt{inversePowerMethod} 函数用于通过反幂法求解矩阵 $A$ 的特征向量。

\textbf{使用方式:} 调用 \texttt{inversePowerMethod(matrix, lambda)} 函数,传入矩阵 $A$ 和特征值 $\lambda$。函数将返回矩阵的特征向量。

该函数利用反幂法求解矩阵 $A$ 对应于特征值 $\lambda$ 的特征向量。

\section*{\centerline{三. 实验结果}}
%\subsection*{Exercise1}
\centerline{\includegraphics[width=\textwidth]{7.1.2.png}}
\begin{table}[H]
	\centering
	\begin{tabular}{|*{10}{c|}}
		\hline
		\multicolumn{10}{|c|}{$n=50$时矩阵的特征值} \\ \hline
		2.0038 & 2.0152 & 2.0341 & 2.0604 & 2.0941 & 2.1351 & 2.1831 & 2.2380 & 2.2996 & 2.3676 \\ \hline
		2.4418 & 2.5220 & 2.6077 & 2.6988 & 2.7947 & 2.8953 & 3.0000 & 3.1085 & 3.2204 & 3.3353 \\ \hline
		3.4527 & 3.5721 & 3.6932 & 3.8155 & 3.9384 & 4.0616 & 4.1845 & 4.3068 & 4.4279 & 4.5473 \\ \hline
		4.6647 & 4.7796 & 4.8915 & 5.0000 & 5.1047 & 5.2053 & 5.3012 & 5.3923 & 5.4780 & 5.5582 \\ \hline
		5.6324 & 5.7004 & 5.7620 & 5.8169 & 5.8649 & 5.9059 & 5.9396 & 5.9659 & 5.9848 & 5.9962 \\ \hline
		\multicolumn{4}{|c|}{循环次数:6} & \multicolumn{6}{c|}{运行时间: 0.68 seconds.} \\ \hline
	\end{tabular}
\end{table}

\begin{table}[H]
	\centering
	\begin{tabular}{|*{10}{c|}}
		\hline
		\multicolumn{10}{|c|}{$n=60$时矩阵的特征值} \\ \hline
		2.0027 & 2.0106 & 2.0238 & 2.0423 & 2.0659 & 2.0947 & 2.1286 & 2.1674 & 2.2110 & 2.2594 \\ \hline
		2.3124 & 2.3699 & 2.4318 & 2.4977 & 2.5677 & 2.6415 & 2.7188 & 2.7996 & 2.8835 & 2.9704 \\ \hline
		3.0600 & 3.1522 & 3.2465 & 3.3429 & 3.4410 & 3.5406 & 3.6414 & 3.7432 & 3.8456 & 3.9485 \\ \hline
		4.0515 & 4.1544 & 4.2568 & 4.3586 & 4.4594 & 4.5590 & 4.6571 & 4.7535 & 4.8478 & 4.9400 \\ \hline
		5.0296 & 5.1165 & 5.2004 & 5.2812 & 5.3585 & 5.4323 & 5.5023 & 5.5682 & 5.6301 & 5.6876 \\ \hline
		5.7406 & 5.7890 & 5.8326 & 5.8714 & 5.9053 & 5.9341 & 5.9577 & 5.9762 & 5.9894 & 5.9973 \\ \hline
		\multicolumn{4}{|c|}{循环次数:6} & \multicolumn{6}{c|}{运行时间: 1.9510 seconds.} \\ \hline
	\end{tabular}
\end{table}

\begin{table}[H]
	\centering
	\begin{tabular}{|*{10}{c|}}
		\hline
		\multicolumn{10}{|c|}{$n=70$时矩阵的特征值} \\ \hline
		2.0020 & 2.0078 & 2.0176 & 2.0312 & 2.0487 & 2.0701 & 2.0952 & 2.1240 & 2.1565 & 2.1926 \\ \hline
		2.2323 & 2.2754 & 2.3219 & 2.3716 & 2.4246 & 2.4806 & 2.5396 & 2.6015 & 2.6661 & 2.7334 \\ \hline
		2.8031 & 2.8751 & 2.9493 & 3.0257 & 3.1039 & 3.1838 & 3.2654 & 3.3484 & 3.4327 & 3.5181 \\ \hline
		3.6044 & 3.6915 & 3.7792 & 3.8674 & 3.9558 & 4.0442 & 4.1326 & 4.2208 & 4.3085 & 4.3956 \\ \hline
		4.4819 & 4.5673 & 4.6516 & 4.7346 & 4.8162 & 4.8961 & 4.9743 & 5.0507 & 5.1249 & 5.1969 \\ \hline
		5.2666 & 5.3339 & 5.3985 & 5.4604 & 5.5194 & 5.5754 & 5.6284 & 5.6781 & 5.7246 & 5.7677 \\ \hline
		5.8074 & 5.8435 & 5.8760 & 5.9048 & 5.9299 & 5.9513 & 5.9688 & 5.9824 & 5.9922 & 5.9980 \\ \hline
		\multicolumn{4}{|c|}{循环次数:6} & \multicolumn{6}{c|}{运行时间: 3.1850 seconds.} \\ \hline
	\end{tabular}
\end{table}

\begin{table}[H]
	\centering
	\begin{tabular}{|*{10}{c|}}
		\hline
		\multicolumn{10}{|c|}{$n=80$时矩阵的特征值} \\ \hline
		2.0015 & 2.0060 & 2.0135 & 2.0240 & 2.0375 & 2.0539 & 2.0733 & 2.0955 & 2.1206 & 2.1486 \\ \hline
		2.1793 & 2.2127 & 2.2489 & 2.2877 & 2.3290 & 2.3729 & 2.4192 & 2.4679 & 2.5189 & 2.5721 \\ \hline
		2.6275 & 2.6850 & 2.7444 & 2.8057 & 2.8688 & 2.9336 & 3.0000 & 3.0679 & 3.1372 & 3.2078 \\ \hline
		3.2796 & 3.3525 & 3.4264 & 3.5011 & 3.5766 & 3.6527 & 3.7293 & 3.8064 & 3.8837 & 3.9612 \\ \hline
		4.0388 & 4.1163 & 4.1936 & 4.2707 & 4.3473 & 4.4234 & 4.4989 & 4.5736 & 4.6475 & 4.7204 \\ \hline
		4.7922 & 4.8628 & 4.9321 & 5.0000 & 5.0664 & 5.1312 & 5.1943 & 5.2556 & 5.3150 & 5.3725 \\ \hline
		5.4279 & 5.4811 & 5.5321 & 5.5808 & 5.6271 & 5.6710 & 5.7123 & 5.7511 & 5.7873 & 5.8207 \\ \hline
		5.8514 & 5.8794 & 5.9045 & 5.9267 & 5.9461 & 5.9625 & 5.9760 & 5.9865 & 5.9940 & 5.9985 \\ \hline
		\multicolumn{4}{|c|}{循环次数:6} & \multicolumn{6}{c|}{运行时间: 6.3900 seconds.} \\ \hline
	\end{tabular}
\end{table}

\begin{table}[H]
	\centering
	\begin{tabular}{|*{10}{c|}}
		\hline
		\multicolumn{10}{|c|}{$n=90$时矩阵的特征值} \\ \hline
		2.0012 & 2.0048 & 2.0107 & 2.0190 & 2.0297 & 2.0428 & 2.0581 & 2.0758 & 2.0958 & 2.1180 \\ \hline
		2.1425 & 2.1692 & 2.1981 & 2.2291 & 2.2622 & 2.2974 & 2.3347 & 2.3739 & 2.4150 & 2.4581 \\ \hline
		2.5030 & 2.5496 & 2.5980 & 2.6481 & 2.6998 & 2.7530 & 2.8077 & 2.8639 & 2.9214 & 2.9801 \\ \hline
		3.0401 & 3.1013 & 3.1635 & 3.2267 & 3.2908 & 3.3558 & 3.4215 & 3.4879 & 3.5550 & 3.6225 \\ \hline
		3.6905 & 3.7589 & 3.8276 & 3.8965 & 3.9655 & 4.0345 & 4.1035 & 4.1724 & 4.2411 & 4.3095 \\ \hline
		4.3775 & 4.4450 & 4.5121 & 4.5785 & 4.6442 & 4.7092 & 4.7733 & 4.8365 & 4.8987 & 4.9599 \\ \hline
		5.0199 & 5.0786 & 5.1361 & 5.1923 & 5.2470 & 5.3002 & 5.3519 & 5.4020 & 5.4504 & 5.4970 \\ \hline
		5.5419 & 5.5850 & 5.6261 & 5.6653 & 5.7026 & 5.7378 & 5.7709 & 5.8019 & 5.8308 & 5.8575 \\ \hline
		5.8820 & 5.9042 & 5.9242 & 5.9419 & 5.9572 & 5.9703 & 5.9810 & 5.9893 & 5.9952 & 5.9988 \\ \hline
		\multicolumn{4}{|c|}{循环次数:6} & \multicolumn{6}{c|}{运行时间: 12.5440 seconds.} \\ \hline
	\end{tabular}
\end{table}

\begin{table}[H]
	\centering
	\begin{tabular}{|*{10}{c|}}
		\hline
		\multicolumn{10}{|c|}{$n=100$时矩阵的特征值} \\ \hline
		2.0010 & 2.0039 & 2.0087 & 2.0155 & 2.0241 & 2.0347 & 2.0472 & 2.0616 & 2.0779 & 2.0960 \\ \hline
		2.1159 & 2.1377 & 2.1613 & 2.1867 & 2.2138 & 2.2426 & 2.2732 & 2.3054 & 2.3392 & 2.3747 \\ \hline
		2.4117 & 2.4503 & 2.4904 & 2.5319 & 2.5748 & 2.6192 & 2.6648 & 2.7118 & 2.7600 & 2.8094 \\ \hline
		2.8599 & 2.9116 & 2.9643 & 3.0180 & 3.0727 & 3.1282 & 3.1846 & 3.2418 & 3.2997 & 3.3583 \\ \hline
		3.4176 & 3.4774 & 3.5376 & 3.5984 & 3.6595 & 3.7210 & 3.7827 & 3.8446 & 3.9067 & 3.9689 \\ \hline
		4.0311 & 4.0933 & 4.1554 & 4.2173 & 4.2790 & 4.3405 & 4.4016 & 4.4624 & 4.5226 & 4.5824 \\ \hline
		4.6417 & 4.7003 & 4.7582 & 4.8154 & 4.8718 & 4.9273 & 4.9820 & 5.0357 & 5.0884 & 5.1401 \\ \hline
		5.1906 & 5.2400 & 5.2882 & 5.3352 & 5.3808 & 5.4252 & 5.4681 & 5.5096 & 5.5497 & 5.5883 \\ \hline
		5.6253 & 5.6608 & 5.6946 & 5.7268 & 5.7574 & 5.7862 & 5.8133 & 5.8387 & 5.8623 & 5.8841 \\ \hline
		5.9040 & 5.9221 & 5.9384 & 5.9528 & 5.9653 & 5.9759 & 5.9845 & 5.9913 & 5.9961 & 5.9990 \\ \hline
		\multicolumn{4}{|c|}{循环次数:6} & \multicolumn{6}{c|}{运行时间: 23.5510 seconds.} \\ \hline
	\end{tabular}
\end{table}

%\subsection*{Exercise2}

\begin{table}[H]
	\centering
	\begin{tabular}{|*{10}{c|}}
		\hline
		\multicolumn{10}{|c|}{最小特征值对应的特征向量} \\ \hline
		$0.0044$ & $0.0087$ & $0.0131$ & $0.0175$ & $0.0218$ & $0.0261$ & $0.0304$ & $0.0347$ & $0.0389$ & $0.0431$ \\ \hline
		$0.0472$ & $0.0513$ & $0.0554$ & $0.0594$ & $0.0633$ & $0.0672$ & $0.0710$ & $0.0747$ & $0.0784$ & $0.0820$ \\ \hline
		$0.0855$ & $0.0889$ & $0.0923$ & $0.0955$ & $0.0987$ & $0.1018$ & $0.1048$ & $0.1076$ & $0.1104$ & $0.1130$ \\ \hline
		$0.1156$ & $0.1180$ & $0.1204$ & $0.1226$ & $0.1247$ & $0.1266$ & $0.1285$ & $0.1302$ & $0.1318$ & $0.1333$ \\ \hline
		$0.1346$ & $0.1358$ & $0.1369$ & $0.1378$ & $0.1387$ & $0.1393$ & $0.1399$ & $0.1403$ & $0.1406$ & $0.1407$ \\ \hline
		$0.1407$ & $0.1406$ & $0.1403$ & $0.1399$ & $0.1393$ & $0.1387$ & $0.1379$ & $0.1369$ & $0.1358$ & $0.1346$ \\ \hline
		$0.1333$ & $0.1318$ & $0.1302$ & $0.1285$ & $0.1267$ & $0.1247$ & $0.1226$ & $0.1204$ & $0.1181$ & $0.1156$ \\ \hline
		$0.1131$ & $0.1104$ & $0.1077$ & $0.1048$ & $0.1018$ & $0.0988$ & $0.0956$ & $0.0923$ & $0.0890$ & $0.0856$ \\ \hline
		$0.0820$ & $0.0784$ & $0.0748$ & $0.0710$ & $0.0672$ & $0.0633$ & $0.0594$ & $0.0554$ & $0.0513$ & $0.0472$ \\ \hline
		$0.0431$ & $0.0389$ & $0.0347$ & $0.0304$ & $0.0261$ & $0.0218$ & $0.0175$ & $0.0131$ & $0.0088$ & $0.0044$ \\ \hline
		\multicolumn{4}{|c|}{最小特征值为:$0.000967979$} & \multicolumn{2}{c|}{迭代次数为:$2$} & \multicolumn{4}{c|}{运行时间: $0.0650$ seconds.} \\ \hline
	\end{tabular}
\end{table}

\begin{table}[H]
	\centering
	\begin{tabular}{|*{10}{c|}}
		\hline
		\multicolumn{10}{|c|}{最大特征值对应的特征向量} \\ \hline
		$-0.0044$ & $0.0087$ & $-0.0131$ & $0.0175$ & $-0.0218$ & $0.0261$ & $-0.0304$ & $0.0347$ & $-0.0389$ & $0.0431$ \\ \hline
		$-0.0472$ & $0.0513$ & $-0.0554$ & $0.0594$ & $-0.0633$ & $0.0672$ & $-0.0710$ & $0.0747$ & $-0.0784$ & $0.0820$ \\ \hline
		$-0.0855$ & $0.0890$ & $-0.0923$ & $0.0956$ & $-0.0987$ & $0.1018$ & $-0.1048$ & $0.1076$ & $-0.1104$ & $0.1131$ \\ \hline
		$-0.1156$ & $0.1181$ & $-0.1204$ & $0.1226$ & $-0.1247$ & $0.1266$ & $-0.1285$ & $0.1302$ & $-0.1318$ & $0.1333$ \\ \hline
		$-0.1346$ & $0.1358$ & $-0.1369$ & $0.1379$ & $-0.1387$ & $0.1393$ & $-0.1399$ & $0.1403$ & $-0.1406$ & $0.1407$ \\ \hline
		$-0.1407$ & $0.1406$ & $-0.1403$ & $0.1399$ & $-0.1393$ & $0.1387$ & $-0.1379$ & $0.1369$ & $-0.1358$ & $0.1346$ \\ \hline
		$-0.1333$ & $0.1318$ & $-0.1302$ & $0.1285$ & $-0.1266$ & $0.1247$ & $-0.1226$ & $0.1204$ & $-0.1181$ & $0.1156$ \\ \hline
		$-0.1131$ & $0.1104$ & $-0.1076$ & $0.1048$ & $-0.1018$ & $0.0987$ & $-0.0956$ & $0.0923$ & $-0.0890$ & $0.0855$ \\ \hline
		$-0.0820$ & $0.0784$ & $-0.0747$ & $0.0710$ & $-0.0672$ & $0.0633$ & $-0.0594$ & $0.0554$ & $-0.0513$ & $0.0472$ \\ \hline
		$-0.0431$ & $0.0389$ & $-0.0347$ & $0.0304$ & $-0.0261$ & $0.0218$ & $-0.0175$ & $0.0131$ & $-0.0087$ & $0.0044$ \\ \hline
		\multicolumn{4}{|c|}{最大特征值为:$3.9990$} & \multicolumn{2}{c|}{迭代次数为:$3$} & \multicolumn{4}{c|}{运行时间: $0.0760$ seconds.} \\ \hline
	\end{tabular}
\end{table}


\section*{\centerline{四. 结果分析}}

\subsection*{Exercise1}

程序针对不同阶数的实对称三对角阵成功求解了全部特征值。随着阶数的增加,运行时间呈正相关增长,从 0.68 秒到 23.55 秒,这是因为更高阶的矩阵需要更多的计算。另外,迭代次数随阶数增加而保持相对稳定,这表明算法对于不同阶数的矩阵都能以较少的迭代次数达到收敛。

\subsection*{Exercise2}
程序成功利用二分法求解了给定实对称三对角阵的最大和最小特征值,并利用反幂法计算了对应的特征向量。针对 100 阶的矩阵,最小特征值为 0.000967979,并在 2 次迭代后获得。相应的最大特征值为 3.9990,迭代 3 次后得到对应的特征向量。值得注意的是,这两个结果的计算时间都相对较短,分别为 0.0650 秒和 0.0760 秒。这表明算法对于求解这个特定阶数的实对称三对角阵的特征值和特征向量具有较高的效率。

\end{document}